%*******************************************************
% Abstract
%*******************************************************
%\renewcommand{\abstractname}{Abstract}
\pdfbookmark[1]{Abstract}{Abstract}
\begingroup
\let\clearpage\relax
\let\cleardoublepage\relax
\let\cleardoublepage\relax

\chapter*{Abstract}

The Southern Ocean is a major sink for anthropogenic CO$_2$ emissions and hence it plays an essential role in modulating global carbon cycle and climate change. Previous studies based on observations show pronounced decadal variations of carbon uptake in the Southern Ocean in recent decades and this variability is largely driven by internal climate variability. However, due to limited ensemble size of simulations, the variability of this important ocean sink is still poorly assessed by the state-of-the-art earth system models (ESMs). To assess the internal variability of carbon sink in the Southern Ocean, we use a large ensemble of 100 member simulations based on the Max Planck Institute-ESM (MPI-ESM). Here we use model simulations from 1980-2015 to compare with available observation-based dataset. We found several ensemble members showing decadal trends in the carbon sink, which are similar to the trends shown in observations. This result suggests that MPI-ESM large ensemble simulations are able to reproduce decadal variation of carbon sink in the Southern Ocean. Moreover, the trends of Southern Ocean carbon sink in MPI-ESM are mainly contributed by region between 50-60$^\circ$S. %To understand the internal variability of the air-sea carbon fluxes in the Southern Ocean, we further investigate the variability of underlying processes, such as physical climate variability and ocean biological processes. 
Our results focus on the impact of biology on decadal trends of carbon sink. Primary production in area from 50-60$^\circ$S is very sensible to euphotic water column stability. Changes in the physical state of the water column influence biological drawdown of ocean surface pCO$_2$ and hence the Southern Ocean carbon sink.
  
\vfill

\begin{otherlanguage}{ngerman}
\pdfbookmark[1]{Zusammenfassung}{Zusammenfassung}
\chapter*{Zusammenfassung}
Kurze Zusammenfassung des Inhaltes in deutscher Sprache\dots 
\end{otherlanguage}

\endgroup			

\vfill