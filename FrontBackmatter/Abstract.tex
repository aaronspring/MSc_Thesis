%*******************************************************
% Abstract
%*******************************************************
%\renewcommand{\abstractname}{Abstract}
\pdfbookmark[1]{Abstract}{Abstract}
%\begingroup
%\let\clearpage\relax
%\let\cleardoublepage\relax
%\let\cleardoublepage\relax

\chapter*{Abstract}
%The Southern Ocean is a major sink for anthropogenic CO$_2$ emissions and hence it plays an essential role in modulating global carbon cycle and climate change. Previous studies based on observations show pronounced decadal variations of carbon uptake in the Southern Ocean in recent decades and this variability is largely driven by internal climate variability. However, due to limited ensemble size of simulations, the variability of this important ocean sink is still poorly assessed by the state-of-the-art earth system models (ESMs). To assess the internal variability of carbon sink in the Southern Ocean, we use a large ensemble of 100 member simulations based on the Max Planck Institute-ESM (MPI-ESM). Here we use model simulations from 1980-2015 to compare with available observation-based dataset. We found several ensemble members showing decadal trends in the carbon sink, which are similar to the trends shown in observations. This result suggests that MPI-ESM large ensemble simulations are able to reproduce decadal variation of carbon sink in the Southern Ocean. Moreover, the trends of Southern Ocean carbon sink in MPI-ESM are mainly contributed by region between 50-60$^\circ$S. %To understand the internal variability of the sea-air carbon fluxes in the Southern Ocean, we further investigate the variability of underlying processes, such as physical climate variability and ocean biological processes. 
%Our results focus on the impact of biology on decadal trends of carbon sink. Primary production in area from 50-60$^\circ$S is very sensible to euphotic water column stability. Changes in the physical state of the water column influence biological drawdown of ocean surface pCO$_2$ and hence the Southern Ocean carbon sink.
Recent observations suggest pronounced decadal variations in the Southern Ocean carbon sink. However, due to the sparse spatial and temporal coverage of observations, it is challenging to discern the dynamics of internally varying processes. Earth system models (ESMs), while being a useful tool to analyze processes that contribute to variability, rarely capture this variability in a small ensemble size. I assess modeled decadal internal variability by analyzing a large ensemble of 100 historical simulations based on Max Planck Institute's ESM (MPI-ESM) starting from different initial conditions but using identical forcing.

The modeled decadal internal variability of the Southern Ocean carbon sink south of 35$^\circ$S is quantified to be $\sim\pm0.19$ PgC/yr/decade in the historical period. This amounts to $\sim12\%$ of the mean state of carbon sink at $\sim1.15$ PgC/yr and dominates over the forced signal of $\sim-0.14$ PgC/yr/decade. MPI-ESM Large Ensemble captures decadal variations of similar magnitude as suggested by observations. The largest variability is found at 50-60$^\circ$S, where CO$_2$ flux follows two wind-driven regimes: stronger winds enhance the upper-ocean overturning circulation and the corresponding upwelling of deep waters weakens the carbon sink. For weakening winds, the upper-ocean circulation slows down and hence strengthens the carbon sink.



%\vfill
\clearpage

\begin{otherlanguage}{ngerman}
\pdfbookmark[1]{Zusammenfassung}{Zusammenfassung}
\chapter*{Zusammenfassung}
%Neue Beobachtungsdaten zeigen dekadische Schwankungen in der Kohlenstoffsenke im Südlichen Ozean auf. Die spärliche räumliche und temporäre Beobachtungsdatendichte macht das Unterscheiden der Dynamik der variablen Prozesse herausfordernd. Erdsystemmodelle, die ein nützliches Hilfsmittel zum Analysieren von variablen Prozessen sind, erfassen selten diese internen Schwankungen. Durch das Analysieren eines Ensembles mit 100 historischen Simulationen basierend auf dem Max-Planck-Institut ESM (MPI-ESM) mit leicht veränderten Anfangsbedingungen untersuche ich modellierte interne Variabilität. 

%Die modellierte dekadische interne Variabilität der Kohlenstoffsenke im Südlichen Ozean südlich von 35$^\circ$S beträgt $\sim\pm0.19$ PgC/Jahr/Dekade. Dies macht $\sim12\%$ der durchschnittlichen Kohlenstoffaufnahme von $\sim-1.15$ PgC/Jahr aus und dominiert über das erzwungene Klimawandelsignal von $\sim-0.14$ PgC/Jahr/Dekade. Die 100 MPI-ESM Simulationen erfassen dekadische Trends von ähnlicher Stärke im Vergleich mit Beobachtungsdaten. In der Region mit der höchsten Variabilität bei 50-60$^\circ$S ergeben sich zwei wind-getriebene Regime: Stärkere Winde stimulieren die obere Umwelzzirkulation, was das Aufsteigen von Tiefenwasser verstärkt. Das darauf folgene Ausgasen reduziert die Kohlenstoffsenke. Schwächere Wind reduzieren die Umwelzzirkulation und stärken somit die Kohlenstoffsenke.
%// sollte kuerzer als 200 Worte sein

In kürzlich veröffentlichen Beobachtungsdaten lassen sich dekadische Schwankungen in der Kohlenstoffsenke im Südlichen Ozean erkennen. Dabei stellt es sich aufgrund der geringen räumlichen und temporären Beobachtungsdatendichte als schwierig dar, die Dynamik der variablen Prozesse zu unterscheiden. Erdsystemmodelle, die ein nützliches Hilfsmittel zur Analyse von variablen Prozessen sind, erfassen diese internen Schwankungen nur selten. In der vorliegenden Masterarbeit untersuche ich modellierte interne Variabilität mittels Analyse eines Ensembles von 100 historischen Simulationen,
basierend auf dem Max-Planck-Institut ESM (MPI-ESM) mit leicht
veränderten Anfangsbedingungen.

Die modellierte dekadische interne Variabilität der Kohlenstoffsenke im Südlichen Ozean südlich von 35$^\circ$S beträgt $\sim\pm0.19$ PgC/Jahr/ Dekade. Dies macht $\sim12$\% der durchschnittlichen Kohlenstoffaufnahme von $\sim1.15$ PgC/Jahr aus und dominiert über das erzwungene Klimawandelsignal von $\sim -0.14$ PgC/Jahr/Dekade. Die 100 MPI-ESM Simulationen erfassen dekadische Trends von ähnlicher Stärke vergleichbar mit Beobachtungsdaten. Die höchste Variabilität lässt sich in der Region 50-60$^\circ$S erkennen, wo die Kohlenstoffflüsse durch folgende zwei Windregime beeinflusst werden: herrschen stärkere Winde, stimulieren sie die obere Umwelzzirkulation, was das Aufsteigen von Tiefenwasser verstärkt und durch das darauf folgende Ausgasen die Kohlenstoffsenke reduziert; auf der anderen Seite verlangsamen schwächere Winde die Umwelzzirkulation und stärken somit die Kohlenstoffsenke.

\end{otherlanguage}

%\endgroup			

%\vfill