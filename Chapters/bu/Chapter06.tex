%************************************************
%\chapter{Summary and conclusions} % $\mathbb{ZNR}$
\chapter{Conclusions and Outlook} \label{ch:conclusions}% $\mathbb{ZNR}$
%************************************************

%zusammenfassen und heftig verlinken und zitieren
%\paragraph{Summary} %and internal variability value $\pm$X; make sure research questions are answered}
% explicit revisiting research questions
To assess Southern Ocean CO$_2$ flux decadal internal variability, I analyze 100 historical \acf{MPI-ESM} simulations in the historical period from 1980 to 2004, for which the observational sea-air CO$_2$ flux product \acs{SOM-FFN} is available. In conclude this thesis by revisiting the research questions posed in the introduction.\newline

\textit{What is the modeled internal variability of the Southern Ocean carbon sink?}\\

I estimate the modeled decadal internal variability $\sigma_{DIV}=0.22$ PgC/yr (see fig. \ref{fig:evolution_southern_ocean_carbon_sink}, \ref{fig:SOCS_temporal_gaussian}a). Compared to the average 1980-2004 CO$_2$ flux south of 35$^\circ$S of 1.15 PgC/yr, decadal internal variability $\sigma_{DIV}$ amounts to $15\%$ of the absolute Southern Ocean carbon sink. Internal variations dominates over the forced signal of 0.015 PgC/yr by one order of magnitude and hence are crucial in understanding changes in the Southern Ocean carbon sink. The area at 50-60$^\circ$S hold the largest decadal variability (see fig. \ref{fig:SOCS_ensmean_ensstd}b) and dominates the overall Southern Ocean CO$_2$ flux trend (see \autoref{ch:trends}). \acs{MPI-ESM LE} contains decades with CO$_2$ flux trends of similar magnitude and monotony as suggested by observations \citep{landschuetzer2015} (see fig. \ref{fig:heatmap}).\newline

\textit{What are the sources of internal variability?}\\

The variability in strength and position of westerly winds dominates the variability in CO$_2$ flux with two distinct wind-driven regimes (see fig. \ref{fig:scatter}): 

On the one hand, stronger and southward shifting westerly winds associated with a positive trend in the \acf{SAM} reduce the Southern Ocean carbon sink. On the other hand, weakening and northward shifted westerly winds leads to an increase in the Southern Ocean carbon sink.\newline

\textit{What are the contributions of different processes to multi-year trends in sea-air CO$_2$ flux?}\\
%atm sst circ bio
The increasing atmospheric pCO$_{\text{2,atm}}$ drives all sea-air CO$_2$ flux trends alike to a more CO$_2$ uptake state, whereas the oceanic pCO$_{\text{2,ocean}}$ is susceptible to internally varying processes:

For intensifying and southward shifting westerly winds, \acf{SST} cooling decreases pCO$_{\text{2,ocean}}$, which drives CO$_2$ uptake (\autoref{sec:trends_pos_thermal}). Cooling and deeper wind-mixing reduce primary production, which increases pCO$_{\text{2,ocean}}$ drives CO$_2$ outgassing (\autoref{sec:trends_pos_biology}). Theses responses are dominated by a process that cannot be accounted for directly and is likely the increased upper-ocean overturning circulation, which enhances outgassing of older deep waters  and overall weakens the Southern Ocean carbon sink (\autoref{ch:pCO2separation}). 

Vice versa, weaker and northward shifting westerly winds the trends reverse and decreasing upwelling is likely to result in an increase in the Southern Ocean carbon sink. %[how deep should the summary of processes go? now its shallow! should I talk about thermal, bio and circulation explicitly - I already evaluate the processes in the chapter right before!]\newline

\vspace{1.5cm}

%\paragraph{Comparability to observations}
What do we learn from this large ensemble simulation about the Southern Ocean carbon sink? While \acs{MPI-ESM LE} does not aim to reproduce CO$_2$ flux trends suggested by observations in the first place, but it proves perturbed initial conditions large ensemble simulations are capable of capturing decadal internal variations similar to observations; even if this only applies for the most extreme decadal trends.

Forcing \acs{MPI-ESM} with a historical CO$_2$ emissions instead of prescribed pCO$_{\text{2,atm}}$ increases internal variability of the global carbon sink by 25\% \citep{Ilyina2013}. Therefore, internal variability with a fully coupled carbon cycle for the Southern Ocean carbon sink could have a higher $\sigma_{DIV}$. 

The strong trends discussed in this thesis originate in strong changes of the position and strength of Southern hemisphere westerly winds and effect of those on ocean circulation. However, the parametrized eddies in \acs{MPI-ESM LE} might allow deeper mixing to sustain on longer time-scales than the seasonal timescale at which the eddies would counteract those trends \citep{Thompson2011}. Only a variable definition of isopyncal thickness diffusion could parametrize the expected eddy response from high-resolution simulations \citep{Gent2011,Lovenduski2013}, but the general challenge of differing ocean circulation patterns remains and makes a comparison between a high-resolution resolved eddies and low-resolution parametrized eddies impossible \citep{Bryan2014}. This effect of eddies asks for new course-resolution ensemble with variable isopyncal thickness diffusion [as in \cite{Lovenduski2013}] or for high-resolution perturbed initial conditions large ensemble simulations.\newline


%\paragraph{Outlook: large ensemble variability modeling}
The history of perturbed large initial conditions simulation is fairly recent. The attempt to study internal variability with \acs{MPI-ESM LE} gives first insights into internal variability from many realizations simulations. Understand internal varying processes in our climate system might become increasingly important in the case of global CO$_2$ emission reductions, when the CO$_2$ reduction efforts are tracked by measurements and evaluated by scientists and politicians \citep{Hawkins2009,Lovenduski2015,Marotzke2017}. A further interesting project would be the comparison of different perturbed initial conditions large ensembles based on different models, \ie comparing \acs{CESM} \acs{LE}, \acs{GFDL} \acs{LE} and \acs{MPI-ESM LE}.\newline

%\paragraph{Outlook: observational focus on Southern Ocean}  
Climate models can only be as good as our current understanding of the climate system. For the Southern Ocean, however, there is a desperate need for an increasing amount of measurements to understand the Southern Ocean dynamics and its biogeochemical properties. The recent ARGO data and the newly deployed biogeochemical floats currently advance the basis for understanding in the Southern Ocean. Elevating  numbers of in-situ measurements help to overcome the current challenges in Southern Ocean modelling \citep{Sallee2013,Sallee2013a,Jungclaus2013,Haumann2014,Stoessel2015,Haumann2016}.   
%no strong finnish :(