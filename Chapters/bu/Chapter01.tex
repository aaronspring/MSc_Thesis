%\documentclass[class=article,crop=false, multi=true]{standalone}
%% **************************************************************************************************************
\RequirePackage{fix-cm} % fix some latex issues see: http://texdoc.net/texmf-dist/doc/latex/base/fixltx2e.pdf
\documentclass[ twoside,openright,titlepage,numbers=noenddot,headinclude,%1headlines,% letterpaper a4paper
                footinclude=true,cleardoublepage=empty,abstractoff, % <--- obsolete, remove (todo)
                BCOR=5mm,paper=a4,fontsize=11pt,%11pt,a4paper,%
                ngerman,american,%
                ]{scrreprt}


%********************************************************************
% Note: Make all your adjustments in here
%*******************************************************
\input{classicthesis-config}

\usepackage{graphicx}
\usepackage{indentfirst}
\graphicspath{{/home/mpim/m300524/MSc_Thesis/gfx/}}

\newrobustcmd*{\parentexttrack}[1]{%
  \begingroup
  \blx@blxinit
  \blx@setsfcodes
  \blx@bibopenparen#1\blx@bibcloseparen
  \endgroup}

\AtEveryCite{%
  \let\parentext=\parentexttrack%
  \let\bibopenparen=\bibopenbracket%
  \let\bibcloseparen=\bibclosebracket}

%********************************************************************
% Bibliographies
%*******************************************************
%\addbibresource{Bibliography.bib}
%\addbibresource{AMiede_Publications.bib}
%\bibliographystyle{apalike}
%\addbibresource{SouthernOceanCarbonSink.bib}
\bibliography{Paper/SouthernOceanCarbonSink_new.bib}
%\bibliographystyle{abbrvnat}
%********************************************************************
% Hyphenation
%*******************************************************
%\hyphenation{put special hyphenation here}

% ********************************************************************
% GO!GO!GO! MOVE IT!
%*******************************************************
\begin{document}
\frenchspacing
\raggedbottom
\selectlanguage{american} % american ngerman
%\renewcommand*{\bibname}{new name}
%\setbibpreamble{}
\pagenumbering{roman}
\pagestyle{plain}
%********************************************************************
% Frontmatter
%*******************************************************
%\includegraphics[scale=1,page=2]{coffee.pdf}
%*******************************************************
% Titlepage
%*******************************************************
\begin{titlepage}
    % if you want the titlepage to be centered, uncomment and fine-tune the line below (KOMA classes environment)
    \begin{addmargin}[-1cm]{-3cm}
    \renewcommand{\baselinestretch}{2.00}
    \begin{center}
        \Large  

		\myFaculty \\
		\smallskip
		
		\large
		\myUni

        \hfill

        \vfill

    	\medskip
		
        \vfill
		Master thesis \\ 
		in Physics \\ 
		submitted by \\ 
		\myName \\ 
		born in \myBirthplace  \\
		\myThesisYear 
        
%        \vfill                      

    \end{center}  
  \end{addmargin}    

  
\cleardoublepage  
\begin{center}
  \renewcommand{\baselinestretch}{2.00}
  \Large \bfseries%\sffamily
    \myTitle 
    
  \par
  \vfill
  \large\normalfont
  This Master thesis has been carried out by \myName \\
  at the institute of \myDepartment \\
  under the supervision of \\
  \myProf \\%(Frau/Herrn Prof./Priv.-Doz. Name Surname)
  %% additionally insert second supervisor here if carrying out an
  %% external diploma thesis. Reduce vspace in L. 44 accordingly.
  and \\ 
  \mySecondProf  \\ from \\ \mySecondProfInstitution 
\end{center}\par
\vspace{5\baselineskip}

% reset baselinestretch
\renewcommand{\baselinestretch}{1.00}\normalsize  
  
  
  
     
\end{titlepage}   
\cleardoublepage
%*******************************************************
% Abstract
%*******************************************************
%\renewcommand{\abstractname}{Abstract}
\pdfbookmark[1]{Abstract}{Abstract}
\begingroup
\let\clearpage\relax
\let\cleardoublepage\relax
\let\cleardoublepage\relax

\chapter*{Abstract}

The Southern Ocean is a major sink for anthropogenic CO$_2$ emissions and hence it plays an essential role in modulating global carbon cycle and climate change. Previous studies based on observations show pronounced decadal variations of carbon uptake in the Southern Ocean in recent decades and this variability is largely driven by internal climate variability. However, due to limited ensemble size of simulations, the variability of this important ocean sink is still poorly assessed by the state-of-the-art earth system models (ESMs). To assess the internal variability of carbon sink in the Southern Ocean, we use a large ensemble of 100 member simulations based on the Max Planck Institute-ESM (MPI-ESM). Here we use model simulations from 1980-2015 to compare with available observation-based dataset. We found several ensemble members showing decadal trends in the carbon sink, which are similar to the trends shown in observations. This result suggests that MPI-ESM large ensemble simulations are able to reproduce decadal variation of carbon sink in the Southern Ocean. Moreover, the trends of Southern Ocean carbon sink in MPI-ESM are mainly contributed by region between 50-60$^\circ$S. %To understand the internal variability of the air-sea carbon fluxes in the Southern Ocean, we further investigate the variability of underlying processes, such as physical climate variability and ocean biological processes. 
Our results focus on the impact of biology on decadal trends of carbon sink. Primary production in area from 50-60$^\circ$S is very sensible to euphotic water column stability. Changes in the physical state of the water column influence biological drawdown of ocean surface pCO$_2$ and hence the Southern Ocean carbon sink.
  
\vfill

\begin{otherlanguage}{ngerman}
\pdfbookmark[1]{Zusammenfassung}{Zusammenfassung}
\chapter*{Zusammenfassung}
Kurze Zusammenfassung des Inhaltes in deutscher Sprache\dots // sollte kuerzer als 200 Worte sein
\end{otherlanguage}

\endgroup			

\vfill
\pagestyle{scrheadings}
\cleardoublepage
\include{FrontBackmatter/Contents}
%********************************************************************
% Mainmatter
%*******************************************************
\cleardoublepage
\pagenumbering{arabic}
%\setcounter{page}{90}
% use \cleardoublepage here to avoid problems with pdfbookmark
\cleardoublepage
%\part{Some Kind of Manual}
%\documentclass[class=article,crop=false, multi=true]{standalone}
%% **************************************************************************************************************
\RequirePackage{fix-cm} % fix some latex issues see: http://texdoc.net/texmf-dist/doc/latex/base/fixltx2e.pdf
\documentclass[ twoside,openright,titlepage,numbers=noenddot,headinclude,%1headlines,% letterpaper a4paper
                footinclude=true,cleardoublepage=empty,abstractoff, % <--- obsolete, remove (todo)
                BCOR=5mm,paper=a4,fontsize=11pt,%11pt,a4paper,%
                ngerman,american,%
                ]{scrreprt}


%********************************************************************
% Note: Make all your adjustments in here
%*******************************************************
\input{classicthesis-config}

\usepackage{graphicx}
\usepackage{indentfirst}
\graphicspath{{/home/mpim/m300524/MSc_Thesis/gfx/}}

\newrobustcmd*{\parentexttrack}[1]{%
  \begingroup
  \blx@blxinit
  \blx@setsfcodes
  \blx@bibopenparen#1\blx@bibcloseparen
  \endgroup}

\AtEveryCite{%
  \let\parentext=\parentexttrack%
  \let\bibopenparen=\bibopenbracket%
  \let\bibcloseparen=\bibclosebracket}

%********************************************************************
% Bibliographies
%*******************************************************
%\addbibresource{Bibliography.bib}
%\addbibresource{AMiede_Publications.bib}
%\bibliographystyle{apalike}
%\addbibresource{SouthernOceanCarbonSink.bib}
\bibliography{Paper/SouthernOceanCarbonSink_new.bib}
%\bibliographystyle{abbrvnat}
%********************************************************************
% Hyphenation
%*******************************************************
%\hyphenation{put special hyphenation here}

% ********************************************************************
% GO!GO!GO! MOVE IT!
%*******************************************************
\begin{document}
\frenchspacing
\raggedbottom
\selectlanguage{american} % american ngerman
%\renewcommand*{\bibname}{new name}
%\setbibpreamble{}
\pagenumbering{roman}
\pagestyle{plain}
%********************************************************************
% Frontmatter
%*******************************************************
%\includegraphics[scale=1,page=2]{coffee.pdf}
\include{FrontBackmatter/Titlepage}
\cleardoublepage
\include{FrontBackmatter/Abstract}
\pagestyle{scrheadings}
\cleardoublepage
\include{FrontBackmatter/Contents}
%********************************************************************
% Mainmatter
%*******************************************************
\cleardoublepage
\pagenumbering{arabic}
%\setcounter{page}{90}
% use \cleardoublepage here to avoid problems with pdfbookmark
\cleardoublepage
%\part{Some Kind of Manual}
\include{Chapters/Chapter01}
%\cleardoublepage
%\ctparttext{You can put some informational part preamble text here.}
%\part{The Showcase}
\include{Chapters/Chapter02}
%\addtocontents{toc}{\protect\clearpage} % <--- just debug stuff, ignore
\include{Chapters/Chapter03}
\include{Chapters/Chapter04}
\include{Chapters/Chapter05}
\include{Chapters/Chapter06}

%\include{multiToC} % <--- just debug stuff, ignore for your documents
% ********************************************************************
% Backmatter
%*******************************************************
\appendix
%\renewcommand{\thechapter}{\alph{chapter}}
\cleardoublepage
%\part{Appendix}
\include{Chapters/Chapter0A}
\cleardoublepage\include{FrontBackmatter/ListofFiguresAndTables}
%********************************************************************
% Other Stuff in the Back
%*******************************************************
\cleardoublepage\include{FrontBackmatter/Bibliography}
\cleardoublepage\include{FrontBackmatter/Acknowledgments}
\cleardoublepage\include{FrontBackmatter/Declaration}  
%*******************************************************
\end{document}
% ********************************************************************

%\input{../classicthesis-config}

%\begin{document}

%************************************************
\chapter{Introduction}\label{ch:introduction}
%************************************************
%where to cite Landschuetzer2016

%\paragraph{Why the Southern Ocean is important} 
%The global Carbon cycle


The increasing atmospheric concentrations of carbon dioxide (CO$_2$) drive anthropogenic climate change, which alters the Earth's climate system at a soaring rate \citep{Crutzen2002}. While roughly half of the anthropogenic CO$_2$ emissions remain in the atmosphere, the carbon reservoirs in oceans and on land modulate the global carbon cycle and hence the climate \citep{Quere2016}. The oceans currently take up about 25-30\% of the anthropogenic carbon emissions \citep{Sabine2004}. As a key region, the Southern Ocean is estimated to contribute about 50\% to the global ocean carbon sink \citep{Takahashi2012}. The large uptake ability of the Southern Ocean arises from its unique position in the global overturning circulation \citep{Talley2013}. Deep waters, which have not been exposed to the atmosphere of higher CO$_2$, take up additional anthropogenic CO$_2$ and sink back into the ocean as intermediate waters \citep{Morrison2015}. Additionally, pronounced primary production in the Southern Ocean sequesters carbon into the deep ocean \citep{Falkowski1998}. Moreover, high solubility of pCO$_2$ in the Southern Ocean due to low sea-surface temperature permits the waters to hold higher oceanic pCO$_2$ concentrations than in the tropical oceans \citep{Heinze2015}.\newline %umscchreiben der prozesse
 
The extreme conditions of the Southern Ocean result in sparse spatial and temporal coverage of observations, so even various obser-vation-based CO$_2$ flux products yield large uncertainties \citep{Roedenbeck2015}. Also modeling results have a large spread \citep{Wang2016} but claim the Southern Ocean as a constraint to reduce model uncertainties in future projections \citep{Kessler2016}.
%\paragraph{Southern Ocean observations and demand for models}

Recent observations suggest pronounced decadal variations in the Southern Ocean carbon sink \citep{Roedenbeck2013,landschuetzer2015}. Understanding internal variability is important to recognize whether the climate changes due to external forcing or natural fluctuations of the earth system \citep{Murphy2004}. Contrasting the expected increasing Southern Ocean carbon sink due to the increasing atmospheric forcing, \citeauthor{landschuetzer2015} [{\color{RoyalBlue}2015}] even reported a decadal declining trend. %why is int var so important
Due to sparse measurement data, it is challenging to discern the dynamics of internally varying processes, which demands for the evaluation and can be provided by models. Models, numerical representations of the climate system based on physical, biological and chemical principles, are a useful tool to analyze processes that contribute to variability.\newline

%processes
%The \ac{SAM}, characterizing the strength and position of the westerly winds, is known to be the dominant mode of climate variability in the Southern hemisphere [\cite{Thompson2000}; {\color{RoyalBlue}2011}]. 
Supposing the strength and position of the westerly winds as the major reason for climate variability for the Southern Ocean [\citeauthor{Thompson2011},  {\color{RoyalBlue}2000}; {\color{RoyalBlue}2011}], the carbon cycle responds in the thermal pCO$_2$ effect, advection of carbon and biological production.\newline

By forcing an ocean model with atmospheric reanalysis data, \citeauthor{Lovenduski2007} [{\color{RoyalBlue}2007}; {\color{RoyalBlue}2008}] demonstrated that increased upwelling due to stronger and southward shifted westerly winds in the Southern hemisphere cause a decline in the Southern Ocean carbon sink. %\paragraph{Working hypothesis}
%The \acf{SAM}, characterizing the strength and position of the westerly winds, is known to be the dominant mode of climate variability in the Southern hemisphere \citep{Thompson2000,Thompson2011}. Supposing \acs{SAM} as the major reason for climate variability for the Southern Ocean, how does the carbon system respond? Changes in westerly winds alter circulation patterns, which directly effect the carbon sink via the thermal pCO$_2$ effect, circulation of carbon and biological production.\newline

Yet, \acp{ESM} containing a freely evolving coupled atmospheric and ocean component have challenges in capturing the decadal variations in sea-air CO$_2$ flux as suggest by observations \citep{Wang2016}. Advances in computing power now enable simulating a larger number of model realizations, which increases the likelihood of capturing pronounced decadal trends. Using a large ensemble of simulations with perturbed initial conditions but identical forcing and model allows to separate trends into an ensemble mean trend, the forced signal, and the residual, the internal variability [\cite{McKinley2016}; {\color{RoyalBlue}2017}].\newline

%\paragraph{What I do and research questions}
By using a large ensemble of 100 simulations based on the \acs{MPI-ESM} (hereafter \acs{MPI-ESM LE}) I investigate the decadal internal variability of oceanic carbon uptake to answer the following research questions: 
\begin{itemize}
\item[-] What is the modeled internal variability of the Southern Ocean carbon sink? 
\item[-] What are the sources of this internal variability?
%\item How does variability in biological and physical processes influence the carbon sink?
\item[-] What are the contributions of different processes to multi-year trends in sea-air CO$_2$ flux in the Southern Ocean?
\end{itemize} 

%why did Tatiana want me to do this:
%all coupled models fail to reproduce SO variability, ocean-only models with NCEP forcing catch it, so its probably the winds and need coupled ensemble simulation
\vspace{15mm}
%\paragraph{Revisit processes}
My thesis  revisits the dominant processes leading to strong trends in the Southern Ocean carbon sink in the ocean biogeochemical model \acf{HAMOCC} of \acs{MPI-ESM} [similar to \cite{Lovenduski2007}; {\color{RoyalBlue} 2008}].

This revisit is particularly interesting as other large ensembles of perturbed initial conditions \acs{ESM} simulations do not capture strong decadal variations in the Southern Ocean carbon sink whereas \acs{MPI}{\color{RoyalBlue}-ESM LE} does [private communication N. Lovenduski (\acs{NCAR}) and S. Schlunegger (\acs{GFDL})].\newline%, see section \ref{sec:PICLE} for details].\newline

\vspace{15mm}
Working on my research questions builds the outline for this thesis: 
I describe the large ensemble of \acs{MPI-ESM} simulations in \autoref{ch:methods} and evaluate the model in the key features related to the Southern Ocean carbon sink variability in \autoref{ch:eval}. Hereafter I qualitatively describe the response of individual processes related to changes in Southern Hemisphere winds in \autoref{ch:trends}. %, which are already discussed in the literature, \eg pCO$_2$ temperature effect \citep{Takahashi1993,Lovenduski2007}, circulation \citep{Abernathey2011,Hauck2013,Lauderdale2013,Lovenduski2008} and biology \citep{Lovenduski2005,Hauck2013,Tagliabue2014}. 
In \autoref{ch:pCO2separation} I quantitatively assess the individual responses of different processes on oceanic pCO$_2$.  Finally,  I draw my main conclusions and % on the driving processes for strong multi-year sea-air CO$_2$ flux trends and Understanding the response of the carbon system helps to evaluate the strong trends in \acs{MPI-ESM LE} and 
%Furthermore, I
evaluate how relatable perturbed initial conditions large ensembles are for internal variability in observations and give an outlook on possible future research in \autoref{ch:conclusions}.\newline %obs?!



%\end{document}

\nocite{WilliamsFollows2011}
%\cleardoublepage
%\ctparttext{You can put some informational part preamble text here.}
%\part{The Showcase}
%*****************************************
\chapter{Examples}\label{ch:examples}
%*****************************************
%\setcounter{figure}{10}
% \NoCaseChange{Homo Sapiens}
Ei choro aeterno antiopam mea, labitur bonorum pri no 
\citep{ilyina2013} His no decore
nemore graecis. In eos meis nominavi, liber soluta vim cu. Sea commune
suavitate interpretaris eu, vix eu libris efficiantur.


\section{A New Section}
Illo principalmente su nos. Non message \emph{occidental} angloromanic
da. Debitas effortio simplificate sia se, auxiliar summarios da que,
interlingua se que. Al via multo esser specimen, campo responder que
da. Le usate medical addresses pro, europa origine sanctificate nos
se.

Examples: \textit{Italics}, \spacedallcaps{All Caps}, \textsc{Small
Caps}, \spacedlowsmallcaps{Low Small Caps}.

Acronym testing: \ac{UML} -- \acs{UML} -- \acf{UML} -- \acp{UML}


\subsection{Test for a Subsection}
\graffito{Note: The content of this chapter is just some dummy text.
It is not a real language.}
Lorem ipsum at nusquam appellantur his, ut eos erant homero
concludaturque. Albucius appellantur deterruisset id eam, vivendum


\begin{figure}[bth]
        \myfloatalign
        \subfloat[Asia personas duo.]
        {\includegraphics[width=.45\linewidth]{gfx/example_1}} \quad
        \subfloat[Pan ma signo.]
        {\label{fig:example-b}%
         \includegraphics[width=.45\linewidth]{gfx/example_2}} \\
        \subfloat[Methodicamente o uno.]
        {\includegraphics[width=.45\linewidth]{gfx/example_3}} \quad
        \subfloat[Titulo debitas.]
        {\includegraphics[width=.45\linewidth]{gfx/example_4}}
        \caption[Tu duo titulo debitas latente]{Tu duo titulo debitas
        latente. \ac{DRY}}\label{fig:example}
\end{figure}


%*****************************************
%*****************************************
%*****************************************
%*****************************************
%*****************************************

%\addtocontents{toc}{\protect\clearpage} % <--- just debug stuff, ignore
%************************************************
\chapter{Model evaluation in the Southern Ocean} % $\mathbb{ZNR}$
%************************************************

\label{ch:eval}

%\paragraph{Flow of section - maybe belongs to Introduction last paragraph}
%The Southern Ocean is located on the Southern hemisphere around the Antarctic continent. As the atmospheric waves are disturbed by only few land masses, the Southern ocean carries very zonally symmetric features.

%The variations of oceanic carbon uptake are related to the dynamic background changes in ocean circulation and possibly biology \citep{LeQuere2007}. 
My analysis of the internal variability of the Southern Ocean carbon sink originates in the evaluation of air-sea CO$_2$ flux (\autoref{sec:co2flux_model_eval}). As the strength of the CO$_2$ flux in the Southern Ocean is modulated by the strength of westerly winds \citep{Lovenduski2007}, I assess also the sea-level pressure fields (\autoref{sec:winds_model_eval}) as well as biology (\autoref{sec:biology}) and upper-ocean circulation, which respond to changes in wind (\autoref{sec:UOOC}).

A previous model evaluation of \acs{HAMOCC} for the \ac{CMIP5} provides a global view on biogeochemistry \citep{Ilyina2013}. In this chapter I evaluate the model focusing on the Southern Ocean and specifically the processes related to the carbon sink. In each subsection I compare the modeled mean state of the Southern Ocean to observational data and assess modeled decadal internal variability in spatial distribution and temporal evolution. 


 

\section{CO$_2$ flux}
\label{sec:co2flux_model_eval}

%intro paragraph on CO2flux drivers
The patterns of CO$_2$ flux are mainly controlled by primary production (see in detail \autoref{sec:biology}) and advective transport of \acs{DIC} and alkalinity (see in detail \autoref{sec:UOOC}). Ekman suction, also referred to as upwelling, drives outgassing and Ekman subduction, also referred to as downwelling, drives CO$_2$ uptake or ingassing. Positive values in sea-air CO$_2$ flux indicate outgassing and negative values CO$_2$ uptake by the oceans.\newline %kick because in figure caption?

%\paragraph{Spatial distribution in mean state} 
The climatological ensemble mean state from 1980 to 2004 of the Southern Ocean CO$_2$ flux in \acs{MPI-ESM} marks the Antarctic coastal region is a CO$_2$ sink, whereas the upwelling waters in the Atlantic and Indian sector at 50-60$^\circ$S are outgassing regions and CO$_2$ flux neutral in the Pacific (\autoref{fig:SOCS_ensmean_ensstd-a}). North of 50$^\circ$S, the oceans take up CO$_2$ because of stronger primary production and Ekman subduction.\newline

%The Southern Ocean is - next to coastal regions - the main area for upwelling from deep waters. These carbon-rich waters equilibrate with at the surface with the atmosphere. For this process of outgassing, I use positive values of CO$_2$ flux. Outgassing from upwelling waters predominantly happens at 50-60$^\circ$S in the atlantic and indian ocean sector (fig. \autoref{fig:SOCS_ensmean_ensstd}a). Further north relatively saline Antartic Intermediate Waters (AAIW) get subducted under warmer subtropical waters. This process of downwelling transports anthropogenic carbon into the ocean interior and hence is a carbon uptake region. Near to the Antarctic continent south of 60$^\circ$S, brine injections and surface cooling drives the sinking of waters via open ocean convection, but at the same time sea-ice hinders CO$_2$ flux. 
 
%\paragraph{Spatial mean comparison to observational data}
The observational-based estimate \acs{SOM-FFN} shows similar patterns of carbon uptake and outgassing but in lower absolute values (\autoref{fig:SOCS_ensmean_ensstd-c}). 
Hardly any measurements in the region of seasonal ice cover make a comparison with observation-based values hard to interpret.\newline
%This model evaluation of CO$_2$ flux is very similar to the related comparison of pCO$_2$ with \citep{Takahashi2009}.
%The modeled pCO$_2$ is very similar to the results in \citep{Ilyina2013}.

%\paragraph{Spatial distribution of internal variability} 
For spatial distributions I define decadal internal variability $\sigma_{DIV}$ as the standard deviation over the changes in decades in the whole ensemble (\autoref{sec:DIV}).

The region 45-60$^\circ$S is most variable and characterized by upwelling (\autoref{fig:UOOC_mean-a}) and the southern edge of primary production (\autoref{fig:SO_intpp_ensmean_ensstd-a}). The change in latitudinal position and strength of the westerly winds affects the position and amplitude of upwelling (\autoref{sec:UOOC}) and primary production (\autoref{sec:biology}).\newline

%\paragraph{Spatial variability comparison to observational data}
The observation-based \acs{SOM-FFN} decadal variability $\sigma_{DIV}$ is most pronounced at 50-60$^\circ$S in the Atlantic and Indian and south of 60$^\circ$S in the Pacific \citep{Landschuetzer2016}. The amplitude of decadal variability is smaller than in \acs{MPI-ESM} (\autoref{fig:SOCS_ensmean_ensstd-d}). However, this comparison is limited by large differences in sample size between the 22-year observational record and the 100 \acs{MPI-ESM} realizations.\newline %how relevant is this plot? 
%Reference to CMIP5 mean and std [later in appendix]?



\def\stackalignment{l}



\begin{figure}[bth]
        %\myfloatalign
        \subfloat[\acs{MPI-ESM} ensemble mean CO$_2$ flux \text{      } \text{[kgC m$^{-2}$yr$^{-1}$ /8yrs]}]
        {\label{fig:SOCS_ensmean_ensstd-a}%
       \includegraphics[scale=.95,page=1,trim=7.25cm 15cm 8.2cm 8.4cm,clip]{Overview_SO_co2flux_intpp_ens_t1990s.pdf}} %\quad
       \subfloat[\acs{MPI-ESM} $\sigma_{DIV}$ CO$_2$ flux  \text{      }  \text{      } \text{[kgC m$^{-2}$yr$^{-1}$ /8yrs]}]
        {\label{fig:SOCS_ensmean_ensstd-b}%
       \includegraphics[scale=.95,page=1,trim=13.4cm 15cm 2.25cm 8.4cm,clip]{Overview_SO_co2flux_intpp_ens_t1990s.pdf}} \\
       
        \subfloat[\acs{SOM-FFN} mean CO$_2$ flux \text{      }  \text{      } \text{[kgC m$^{-2}$yr$^{-1}$ /8yrs]}]
        {\label{fig:SOCS_ensmean_ensstd-c}%
         \includegraphics[scale=.95,page=2,trim=7.25cm 15cm 8.2cm 8.4cm,clip]{Overview_SO_co2flux_intpp_ens_t1990s.pdf}} 
        \subfloat[\acs{SOM-FFN} $\sigma_{DIV}$ CO$_2$ flux \text{      }  \text{      } \text{[kgC m$^{-2}$yr$^{-1}$ /8yrs]}]
        {\label{fig:SOCS_ensmean_ensstd-d}%
         \includegraphics[scale=.95,page=2,trim=13.4cm 15cm 2.25cm 8.4cm,clip]{Overview_SO_co2flux_intpp_ens_t1990s.pdf}} 
        \caption{Spatial distribution of the climatology (a,c) and decadal internal variability $\sigma_{DIV}$ (b,d) from 1980-2004 of the Southern Ocean sea-air CO$_2$ flux: (a) \acs{MPI-ESM LE} ensemble mean as forced signal, (b) ensemble decadal standard deviation as decadal internal variability $\sigma_{DIV}$, (c) \acs{SOM-FFN} climatology 1982-2004, (d) \acs{SOM-FFN} decadal variability $\sigma_{DIV}$; negative values indicate ocean uptake.} \label{fig:SOCS_ensmean_ensstd}
\end{figure}

\clearpage
\begin{figure}[hbt]
        \myfloatalign
        \captionsetup[subfigure]{labelformat=empty,justification=centering}
        \subfloat[Annual sea-Air CO$_2$ flux anomaly \text{[PgC/yr]}]
        {\includegraphics[scale=.48,angle=90,page=2,trim=4cm 1.2cm 4.5cm 2.5cm,clip]{co2flux_SO_timeseries_ymjm_35S_1980_2015_trend_8}} 
        \caption{Temporal evolution of the Southern Ocean sea-air CO$_2$ flux anomaly south of 35$^\circ$S with respect to the 1980s. Grey lines show the 100 ensemble members, the black line the ensemble mean, the blue shading is the ensemble decadal internal variability $\sigma_{DIV}$, the red line shows the most  positive 8-year trend, the blue line shows the most negative 8-year trend, the green line represents the \acs{SOM-FFN} observation-based estimate \citep{landschuetzer2015}; negative values indicate carbon uptake; \acs{MPI-ESM LE} is extended under \acs{RCP}4.5 forcing from 2005 to 2015.}\label{fig:evolution_southern_ocean_carbon_sink}
\end{figure}

%\paragraph{Temporal evolution of the mean state and internal variability}
The temporal evolution of the Southern Ocean ensemble mean CO$_2$ flux south of 35$^\circ$S is dominated by the forced negative trend of rising atmospheric CO$_2$ concentrations (\autoref{fig:evolution_southern_ocean_carbon_sink}). Still, no individual realization follows the negative trend strictly monotonic - they oscillate around the ensemble mean state triggered by internal variability. Exemplary, the most extreme monotonic positive and negative 8-year trend is highlighted. The modeled ensemble mean sea-air CO$_2$ flux over the historical period 1980-2004 is $\sim-1.15$ PgC/yr and decadal internal variability $\sigma_{DIV}$ is $\sim\pm0.19$ PgC/yr/decade, larger than the forced signal of $\sim-0.14$ PgC/yr/decade, which reflects the mean of decadal changes. {\color{RoyalBlue}Tab.} \ref{tab:trends_co2flux} shows an overview the statistical quantities related to Southern Ocean carbon sink variability.\newline

%\paragraph{Temporal comparison to observational data} 
The observational-based estimate for the whole region south of 35$^\circ$S shows a strong positive CO$_2$ flux trend in the 1990s and a strong negative trend reinvigorating the carbon sink in the 2000s \citep{landschuetzer2015} (\autoref{fig:evolution_southern_ocean_carbon_sink}). The modeled $\sigma_{DIV}$ is lower than of decadal variations in \acs{SOM-FFN} (\autoref{tab:trends_co2flux}). This means that on multi-year time-scales \acs{SOM-FFN} is 50\% more variable than \acs{MPI-ESM LE}. This comparison must be ingested with caution as influent datasets differ in ensemble-space. \acs{SOM-FFN} would rank as one exceptionally variable 22-year realization into \acs{MPI-ESM LE}.\newline

\begin{table}[hbt]
%\centering
\small
\begin{tabular}{lcccc}
\multicolumn{5}{l} {trend length [TL]} \hspace{.1cm} {8-yr trend} \hspace{1.4cm} {10-yr trend} \hspace{1.cm} \\ \hline
%{8-yr trend} \multicolumn{2}{l} 50-60$^\circ$S \multicolumn{2}{l} 40-50$^\circ$S} \\ \hline
  dataset & \footnotesize \acs{MPI-ESM}  & \footnotesize\acs{SOM-FFN} & \footnotesize\acs{MPI-ESM}  &\footnotesize \acs{SOM-FFN}  \\
  \hline
  mean state [PgC/yr]					& -1.15 & -0.87 & -1.15 & -0.87 \\
  $\sigma_{DIV}$ [PgC/yr/TL]			& $\pm$0.22 & $\pm$0.35 & $\pm$0.19 & $\pm$0.30 \\
  forced signal [PgC/yr/TL] 			& -0.11 & / & -0.14 & / \\
  \hline \hline
  positive trend [PgC/yr/TL]			& +0.70 & +0.38 & +0.53 & +0.39 \\
  corrected trend [PgC/yr/TL]			& +0.81 & +0.49 & +0.67 & +0.53 \\
  probability [\%]						& 0.01  &	20	&  0.1	&	10	\\
  \hline
  negative trend [PgC/yr/TL]			& -0.78 & -0.53 & -0.74 & -0.66 \\
  corrected trend [PgC/yr/TL]			& -0.67 & -0.39 & -0.60 & -0.52 \\
  probability [\%]						&  0.3	&	25	&	0.3	&	10	
\end{tabular}
\caption{Overview of statistical quantities in two trend lengths [TL] in sea-air CO$_2$ flux in the Southern Ocean south of 35$^\circ$S from \acf{MPI-ESM LE} and \acf{SOM-FFN} \citep{landschuetzer2015}; positive and negative trends are the most extreme trends in each dataset; the forced signal is subtracted off the corrected trends; probabilities quantify the likelihood that corrected trends are due to internal variability; 8-yr trends are required to be monotonic.}
\label{tab:trends_co2flux}
\end{table}


The decadal variability $\sigma_{DIV}$ in \acs{SOM-FFN} data is $\sim0.30$ PgC/yr/ decade, with the most positive CO$_2$ flux trend in the 1990s of $\sim+0.39$ PgC/yr/decade and the most negative trend of $\sim-0.66$ PgC/yr/ decade. Compared to \acs{SOM-FFN}'s $\sigma_{DIV}$, both observation-based trends corrected for the modeled forced signal get an attribution to internal variability of $\sim10$\% ($\sim1.6\sigma$). Under the context of \acs{MPI-ESM LE}'s $\sigma_{DIV}$ these magnitudes in observation-based trends could be due to internal variability with probabilities of 1\% ($\sim2.6\sigma$) for 1990s and 2000s trends respectively. This outlines the low likelihood of these \acs{SOM-FFN} CO$_2$ flux trends attributed to internal variability from the perspective of numerical modeling with perturbed initial conditions (\autoref{tab:trends_co2flux}).

As \acs{MPI-ESM LE} simulations are forced with prescribed atmospheric CO$_2$ concentrations, the modeled internal variability of an interactive carbon cycle would be a more realistic counterpart for the observed internal variability. The non-linear changes in atmospheric CO$_2$ concentrations due to the interplay of the oceanic and terrestrial carbon cycle result in 25\% higher internal variability \citep{Ilyina2013}.\newline



%\paragraph{Choice of trends} 
%To analyze the processes driving this decadal internal variability $\sigma_{DIV}$, I focus on individual ensemble realizations. Here, I have to make a compromise between signal strength and robustness versus trendlength. It is a trade-off between decadal and inter-annual variability. The longer a period, the more likely the trend deviates from a monotonic behavior. Therefore longer trends show a less strong signal per trendlength (Fig. \autoref{fig:heatmap}). Longer trends also experience a stronger influence of atmospheric forced trend. I decided to analyze 8-year trends, because they are still very close to a decade, but still they are able to show similar magnitude and monotonic behavior as the observation-based estimate. 


%\paragraph{Highlighted monotonic CO$_2$ flux trends}
%I select the most extreme positive and negative monotonic trends. 
Applying the selection criteria from \autoref{sec:choicetrend} I focus on the two highlighted extreme CO$_2$ flux trends with monotonic behavior (\autoref{fig:evolution_southern_ocean_carbon_sink}). For decadal 10-year trends the most extreme cases are of the same magnitude ($\sim+0.5$ PgC/yr/decade and $\sim-0.7$ PgC/yr/decade) as those derived from observation-based estimates in the 1990s and 2000s \citep{landschuetzer2015} (\autoref{fig:heatmap}). The processes related to decadal and multi-year trends only differ in monotony. Longer trends are more strongly exposed to the forced trend and likely dominated strong short-term varibility and hence do not persist as long. In the following I always also analyze the drivers of two distinct 8-year trends. The most positive monotonic 8-year air-sea CO$_2$ flux trend from 1985 to 1992 in ensemble 78 reaches $\sim+0.5$ PgC/yr/8yrs and the most negative monotonic 8-year trend from 1995 to 2002 in ensemble 143 reaches $\sim-0.6$ PgC/yr/8yrs. 
%Analyzing a range of trend lengths from 6 to 14 years, I assess that \acs{MPI-ESM LE} is able to capture positive and negative multi-year and decadal CO$_2$ flux trends in the Southern Ocean (\autoref{fig:heatmap}). 

%I choose extreme monotonic trends to study the processes driving internal variability in the carbon cycle because stronger trends will give me a stronger signal in the underlying processes, which are discussed in chapter \autoref{ch:trends}.\newline %Therefore, I choose to analyze 8-year trends which are a compromise between trendlength (supposed to be 10 year for decadal variability) and a clearer signal (decays over time).

%The positive CO$_2$ flux trend highlighted in red is 0.7 PgC/8yrs, the negative CO$_2$ flux trend highlighted in blue is -0.8 PgC/8yrs. Those trends are higher than the corresponding trends in the 1990s and 2000s from SOM-FFN, but they that's also not the point of my study.  


%\paragraph{Comparison of ensemble CO$_2$ flux internal variability} This is in contrast to other perturbed inital conditions large ensembles (see section \autoref{sec:PICLE}). Neither \acs{CESM} \acs{LE} nor \acs{GFDL} \acs{LE} can reproduce similar strong decadal CO$_2$ flux trends in the Southern Ocean [personal communication Nikki Lovenduski, Sarah Schlunegger]. \cite{McKinley2016} mentioned a weak carbon sink for \acs{CESM} \acs{LE}. \cite{Resplandy2015} showed that \acs{MPI-ESM}'s dominant area of internal variability is the Southern Ocean, whereas \acs{CESM} is most variable in the tropical and Indian pacific and \acs{GFDL} has large internal variability in tropical Indian pacific as well as the Southern Ocean.

%The different method of initialization could not have caused the behavior of trends, because the CO$_2$ flux pathways of different realizations already cross after few years and upper ocean loses his history after 50 years. The internal variability specific to the model itself is dominating \citep{Lovenduski2016}.  







\clearpage
\section{Winds}
\label{sec:winds_model_eval}
%\paragraph{Spatial distribution in the mean state}
%The Southern Ocean is located on the Southern hemisphere around the Antarctic continent. As the atmospheric waves are disturbed by only few land masses and the insolation changes latitudinal, the Southern ocean carries very zonally symmetric features.
Winds are governed by distributions in \ac{SLP}. They point from high pressure to low pressure systems, but get diverted to their left in the Southern Hemisphere by the Coriolis force. Therefore, strong westerly winds are established between the low-pressure system south of 45$^\circ$S and higher pressure systems over the subtropical gyres (\autoref{fig:SO_winds_ensmean_ensstd-a}). This zonally symmetric \acs{SLP} pattern in the ensemble climatological mean leads to zonally symmetric westerly winds peaking at 50$^\circ$S.\newline

%\paragraph{Spatial mean comparison to observational data}
The spatial zonal distribution in observational data of a reanalysis climatology from \acf{NCEP} is so similar to the \acs{MPI-ESM LE} climatology that I plot the difference between \acs{MPI-ESM LE} climatology and \acs{NCEP} reanalysis \citep{Kalnay1996} (\autoref{fig:SO_winds_ensmean_ensstd-c}). The position of the jets in \acs{ECHAM} is shifted to lower latitudes \citep{Stevens2013} which seems to be a common atmospheric modeling challenge \citep{Kidston2010}. Therefore, the modeled westerlies are too strong from 30-60$^\circ$S and too weak south of 60$^\circ$S. The zonal wind speed difference peaks at 45$^\circ$S at $+1$m/s.

The decadal internal variability $\sigma_{DIV}$ increases zonally towards lower latitudes with the highest internal decadal variability in the pacific sector in the Southern Ocean seasonal ice-covered area (\autoref{fig:SO_winds_ensmean_ensstd-b}).\newline

%\paragraph{Spatial internal variability comparison to observational data}
Observational reanalysis data reveal the same spatial distribution with a higher amplitude of decadal internal variability (\autoref{fig:SO_winds_ensmean_ensstd-d}). The pacific sector reflects the area of an Antarctic dipole induced from the \ac{ENSO}, which manifests itself in variability of sea-ice \citep{Yuan2004}. This is also known as the Pacific-South American (PSA) Oscillation \citep{Sallee2008}. The atmospheric Southern Ocean jet splits and in El Ni\~{n}o conditions intensifies on a northern route  inducing a low pressure system with more storms. La Ni\~{n}a conditions bring a high pressure system with less storms vice versa. Such tele-connections of the tropics into the Southern Ocean are subject of current research and have been rarely analyzed yet for the Southern Ocean carbon sink.\newline

%The direction of winds is lost in the calculation of standard deviations, therefore the wind arrows in fig. \autoref{fig:SO_winds_ensmean_ensstd}b,d only show the magnitude of the internal variability of winds which is highest in the pacific sector of the Southern Ocean and generally decreases with latitude.

\begin{figure}[hbt]
        %\myfloatalign
        \subfloat[\acs{MPI-ESM LE} mean \acs{SLP} \text{[hPa/8yrs]}]
        {\label{fig:SO_winds_ensmean_ensstd-a}%
       \includegraphics[scale=1,page=1,trim=7.25cm 16.5cm 8.2cm 7cm,clip]{Overview_SO_slp_ens_t1990s.pdf}} %\quad
       \subfloat[\acs{MPI-ESM} \acs{SLP} $\sigma_{DIV}$ \text{[hPa/8yrs]} \text{      }  \text{      } \text{      }  \text{      }]
        {\label{fig:SO_winds_ensmean_ensstd-b}%
       \includegraphics[scale=1.0,page=1,trim=13.4cm 16.5cm 2.5cm 7cm,clip]{Overview_SO_slp_ens_t1990s.pdf}} \\
       
        \subfloat[\acs{MPI-ESM}-\acs{NCEP} mean \acs{SLP} \text{[hPa/8yrs]}]
        {\label{fig:SO_winds_ensmean_ensstd-c}%
         \includegraphics[scale=1.,page=1,trim=7.35cm 10.65cm 8.2cm 12.6cm,clip]{Overview_SO_slp_ens_t1990s.pdf}} 
        \subfloat[\acs{NCEP} \acs{SLP} $\sigma_{DIV}$ \text{[hPa/8yrs]}]
        {\label{fig:SO_winds_ensmean_ensstd-d}%
         \includegraphics[scale=1.,page=1,trim=13.4cm 10.65cm 2.5cm 12.6cm,clip]{Overview_SO_slp_ens_t1990s.pdf}} 
        \caption{Spatial distribution of the Southern Ocean \acf{SLP} and wind vectors overlain as arrows: (a) \acs{MPI-ESM} ensemble mean climatology from 1980 to 2004 as forced signal, (b) ensemble decadal standard deviation as decadal internal variability $\sigma_{DIV}$; (c) difference between \acs{MPI-ESM} and reanalysis data from \acs{NCEP} reanalysis climatology \citep{Kalnay1996}, (d) decadal internal variability $\sigma_{DIV}$ from \acs{NCEP} reanalysis climatology.} \label{fig:SO_winds_ensmean_ensstd}
\end{figure}


%\paragraph{Temporal evolution of the mean state and internal variability}
The temporal evolution and internal variability of the annual \ac{SAM} index calculated according to \citep{Gong1999} (\autoref{sec:sam}) is shown in \autoref{fig:SO_winds_ensmean_ensstd}. Positive \acs{SAM} index values are associated with an anomalously low \acs{SLP} over Antarctica that result in a southward-shift and intensification of westerly winds. The \acs{MPI-ESM} ensemble mean has a positive trend as a consequence to anthropogenic CO$_2$ emissions and ozone depletion over Antarctica \citep{Thompson2011}.\newline


\begin{figure}[hbt]
        \myfloatalign
        \captionsetup[subfigure]{labelformat=empty,justification=centering}
        \subfloat[\acf{SAM} index \text{[1]}]
        {\includegraphics[scale=.48,angle=90,page=4,trim=4cm 1.2cm 4.5cm 2.5cm,clip]{co2flux_SO_timeseries_ymjm_35S_1980_2015_trend_8}} 
        \caption{Temporal evolution of the annual \acf{SAM} index according to \citep{Gong1999}. Grey lines show the 100 ensemble members; the black line the ensemble mean; the blue shading is the decadal internal variability $\sigma_{DIV}$; the red line reprensents the SAM during positive sea-air CO$_2$ flux trend; the blue line during the negative CO$_2$ flux trend; the green line shows the station-based \acs{SAM} from \cite{Marshall2003}.}\label{fig:evolution_SAM}
\end{figure}



%\paragraph{Temporal comparison to observational data} 
\acs{MPI-ESM LE} \ac{SAM} index lies in the range of the oberservational station-based \acs{SAM} index from \citep{Marshall2003}, although it follows a weaker trend in the ensemble mean.\newline

%\paragraph{Temporal evolution of extreme carbon trend members}
The positive CO$_2$ flux trend shows a positive trend in \acs{SAM}. Those strengthening winds increase upwelling (\autoref{sec:UOOC}), which brings over-saturated waters to the surface and hence leads to outgassing anomalies. This weakens the carbon sink and vice versa strengthens the carbon sink under the context of weakening westerly winds. The detailed response of primary production and advective transports under the context of changing winds is discussed in detail later in \autoref{ch:trends}.



\clearpage
\section{Biology}
\label{sec:biology}

%\paragraph{Spatial distribution in the mean state} %going from coast to north
%(fig. \autoref{fig:SOCS_ensmean_ensstd}a) 
The strong seasonal cycle in insulation and the sparseness of land topography leads to first-order zonally symmetric constraints for light and \ac{SST}. The high-latitude Southern Ocean is a so-called high nutrient low chlorophyll (HNLC) region, where light is the dominant limiting factor for the low biological production but nutrients are plenty \citep{Falkowski1998}. Summer sea-ice cover and sub-zero \acs{SST} values constrain plankton growth and is responsible for the tiny amount of primary production in the coastal areas as well as the Ross Sea and Weddell Sea. 

With decreasing latitude in the Southern Ocean, primary production increases along with increasing temperatures and light availability. Nutrients are upwelled by the upper-ocean overturning circulation and advected northwards by Ekman transport. This latitudinal increase of primary production peaks at 40-50$^\circ$S, where nutrients are abundant from upwelling and Ekman transport and higher temperatures and light availability foster phytoplankton growth rates. The mixing with warm subtropical waters off the Argentinian coast increases \acs{SST} and leads to a maximum primary production in the Southern Ocean \citep{Behrenfeld2014}. Downstream the Drake passage, the polar front with its cold waters extends more northward \citep{Orsi1995}. Along with lower nutrient concentrations due to increased precipitation from storms it explains the relatively low primary production in the Atlantic sector compared to other longitudinal counterparts. At the subtropical front decreasing nutrient concentrations limit primary production \citep{Behrenfeld2014} (\autoref{fig:SO_intpp_ensmean_ensstd}).\newline


%\paragraph{Spatial mean comparison to observational data}
To evaluate \acs{HAMOCC}'s ability to model the Southern Ocean, I do not compare modeled primary production to chlorophyll-a  concentration derived from satellite data, because satellite images are frequently  hidden by clouds. Instead, I compare the distributions of nitrate which is the limiting nutrient for biological production in this \acs{HAMOCC} version. %nitrate would be the limiting one
The distribution of nitrate shows a strong gradient in \acs{HAMOCC} as well as in the \ac{WOA} 2013  \citep{WOA2013} along the fronts from plentiful of nutrients in the high-latitudes to nutrient depletion in the subtropical gyres (\autoref{fig:SO_comp_nitrate}). In higher latitudes \acs{HAMOCC} underestimates the phosphate concentration by 25\%. These lower nutrient concentrations can be a sign of more nutrient consumption at higher primary production or be the reason for lower primary production. The distribution of another important nutrient phosphate shows a similar spatial pattern (\autoref{fig:SOCS_comp_phosph}).\newline

%\paragraph{Spatial distribution of internal variability} 
Internal decadal variability in vertically integrated primary production in the Southern Ocean is higher in high productivity areas (\autoref{fig:SO_intpp_ensmean_ensstd-b}). The whole region at 45-60$^\circ$S, especially in the Indian sector, shows a enormously high decadal internal variability $\sigma_{DIV}$ relative to the ensemble mean state. Internal variability in the Southern Hemisphere is mostly driven by westerly winds. The explicit effect of those on \acs{HAMOCC} is discussed in \autoref{ch:trends}.\newline


\begin{figure}[bth]
        \myfloatalign
        \subfloat[\acs{MPI-ESM} \acs{INTPP} ensemble mean \text{      }  \text{      } \text{[kgC m$^{-2}$ yr$^{-1}$]}]
        {\label{fig:SO_intpp_ensmean_ensstd-a}%
       \includegraphics[scale=.97,trim=7.25cm 6.3cm 8.2cm 16.8cm,clip]{Overview_SO_co2flux_intpp_ens_t1990s.pdf}} %\quad
        \subfloat[\acs{MPI-ESM} \acs{INTPP} $\sigma_{DIV}$ \text{      }  \text{      } \text{      }  \text{      } \text{[kgC m$^{-2}$ yr$^{-1}$]}]
        {\label{fig:SO_intpp_ensmean_ensstd-b}%
         \includegraphics[scale=.97,trim=13.3cm 6.3cm 2.25cm 16.8cm,clip]{Overview_SO_co2flux_intpp_ens_t1990s.pdf}} \\
        \caption{Spatial distribution of the vertically integrated primary production in the Southern Ocean: (a) climatological \acs{MPI-ESM} ensemble mean from 1980 to 2004 as forced signal and (b) ensemble decadal anomaly standard deviation as decadal internal variability $\sigma_{DIV}$.} \label{fig:SO_intpp_ensmean_ensstd}
\end{figure}

\begin{figure}[bth]
        \myfloatalign
        \subfloat[\acs{MPI-ESM} surface nitrate ensemble mean \text{      }  \text{      } \text{[mmol N m$^{-3}$]}]
        {\label{fig:SO_comp_nitrate-a}%
       \includegraphics[scale=.66,page=3,trim=1.3cm 13.3cm 12.1cm 6.5cm,clip]{Overview_SO_nutrient_comparison.pdf}} %\quad
        \subfloat[\acs{WOA} surface nitrate mean \text{      }  \text{      } \text{      }  \text{      } \text{      }  \text{      }\text{[mmol N m$^{-3}$]}]
        {\label{fig:SO_comp_nitrate-b}%
         \includegraphics[scale=.66,page=3,trim=9.4cm 13.3cm 2.1cm 6.5cm,clip]{Overview_SO_nutrient_comparison.pdf}} \\
        \caption{Spatial distribution of surface nitrate in the Southern Ocean: (a) \acs{MPI-ESM} ensemble mean climatology, (b) \acf{WOA} climatology data \citep{WOA2013}.} \label{fig:SO_comp_nitrate}
\end{figure}

%\paragraph{Temporal evolution of the mean state and internal variability}
Regarding the temporal evolution in the Southern Ocean, primary production is not subject to a strong forced trend in the historical period but varies internally (\autoref{fig:evolution_southern_ocean_intpp}). The weak decreasing trend might be the first sign of increased stratification due to sea-surface warming, which in turn inhibits the mixing of nutrients. But the long-term consequences for primary production is subject to ongoing research and debate \citep{Bopp2013,Taucher2011,Lozier2011,Kessler2016,Krumhardt2017,Deppeler2017}. The decadal internal variability $\sigma_{DIV}$ is 0.5 PgC/yr/decade.

\begin{figure}[bth]
        \myfloatalign
	    \captionsetup[subfigure]{labelformat=empty,justification=centering}        
        \subfloat[\acf{INTPP} \text{[PgC/yr]}]
        {\includegraphics[scale=.48,angle=90,page=3,trim=4cm 1.2cm 4.5cm 2.5cm,clip]{co2flux_SO_timeseries_ymjm_35S_1980_2015_trend_8}} 
		\caption{Temporal evolution of the \acf{INTPP} in the Southern Ocean south of 35$^\circ$S. Grey lines show the 100 ensemble members, the black line the ensemble mean, the blue shading is the decadal internal variability $\sigma_{DIV}$, the red line shows \acs{INTPP} during the positive CO$_2$ flux trend, the blue line during the negative CO$_2$ flux trend.}	\label{fig:evolution_southern_ocean_intpp}
\end{figure}

The absolute amount of primary production exceeds CO$_2$ flux by a factor of $\sim$20. The spatial distribution of export flux at $90m$, which is the particulate organic matter that sinks below the euphotic zone, is the same as for primary production. As a lower vertical boundary for a upper-ocean budget, export flux is of similar magnitude as CO$_2$ flux and thus comparable (\autoref{ch:pCO2separation}).\newline

%\paragraph{Temporal evolution of extreme CO$_2$ flux trends}
The negative CO$_2$ flux trend has a positive trend in primary production, because primary production lowers surface \acs{DIC}, pCO$_{2\text{,ocean}}$ and hence reduces CO$_2$ uptake by the ocean; and vice versa negative trends in primary production lead to positive CO$_2$ flux trends.\newline 

%\paragraph{HAMOCC Southern Ocean performance}
\acs{MPI-ESM} is able to model the general features of the Southern Ocean, such as the characteristics of a high nutrient low chlorophyll region \citep{Bopp2013}. But compared to other models and observational data, the seasonal cycle of phytoplankton blooms is amplified and too early in the Southern Ocean \citep{Bopp2013,Nevison2015,Nevison2016}. The reason of this is under current debate in the MPI biogeochemistry research group. It could involve that the Southern Ocean in \acs{MPI-ESM LE} runs is not iron-limited but while changing the dust input fields and the iron cycling constants lead to an iron-limited Southern Ocean the amplified seasonality is still present [private Communication, Irene Stemmler (MPI, Hamburg)]. Also a look into the grazing of zooplankton and its variable formulation might lead to a more realistic primary production seasonality. Furthermore, the atmospheric bias in westerly winds and the Southern Ocean warm bias \citep{Jungclaus2013} hinders sea-ice to propagate more extensively. A proper representation of Antarctic sea-ice would come along with a cooler, more stratified Southern Ocean which modulates primary production. But the amplified seasonality also occurs in the norwegian \acs{ESM} NorESM which uses \acs{HAMOCC} in a different ocean model, so revisiting seasonality in the high-latitude oceans remains a challenge in \acs{HAMOCC}.

Additionally, the longest standing data records, which are on the northern hemisphere in Iceland \citep{Six1996}, are used for the tuning of free model parameters. Historically, the Southern Ocean has never been in the focus of \acs{HAMOCC}. 









\clearpage

\section{Upper-Ocean Overturning Circulation}
\label{sec:UOOC}

%\paragraph{Spatial distribution in the mean state and internal variability}
The Southern Ocean upper-ocean overturning circulation is driven by the divergence at 40-60$^\circ$S corresponding to strong westerly winds (\autoref{fig:UOOC_mean-a}). 
The isopyncals, which separate water masses, orient themselves at values from \cite{Sallee2013a}, but are shifted to fit to typical depths, which is a common feature in water mass comparison as models have different density biases \citep{Sallee2013a}. 
In the high latitudes south of 50$^\circ$S Ekman pumping brings \ac{CDW} from the ocean interior to the surface. At the surface these waters are transported to the north. At lower latitudes, the surface waters warm and evaporation increases, so relatively cold and low salinity waters known as \ac{AAIW} slide below warmer and more saline \ac{SAMW} to extend northwards at intermediate depth. This process is called downwelling or Ekman subduction.  

The strong upper-ocean overturning circulation in the Southern Hemisphere features upwelling south of 50$^\circ$S, northward transport at 40-60$^\circ$S and downwelling at 30-40$^\circ$S and is known as the Deacon cell \citep{Doeoes1993,Speer2000}. The upper-ocean overturning circulation is driven by the strength and positions of westerly winds. Upwelling steepens and downwelling straightens the isopycnals along which the water masses flow \citep{Marshall2012}. The internal variability in the horizontal processes at intermediate depth is lower than at the surface because the influence of winds decays with depth. The decadal internal variability $\sigma_{DIV}$ of vertical Ekman pumping and subduction is of similar magnitude at 200m and 1000m (\autoref{fig:UOOC_mean}).\newline

%\paragraph{Comparison to observational data}
{\color{RoyalBlue}Fig.} \ref{fig:UOOC_mean-a} is similar to \cite[fig. 1]{DeVries2017}, which uses observational data in an inverse model to demonstrate changes in upper-ocean overturning circulation. Due to different vertical velocity regimes I choose different boundaries for the transport which prevents me from quantitative comparisons. Nevertheless, data of \cite[fig. 1]{DeVries2017} show the main characteristics of the Deacon cell and water transport of comparable magnitude as in \acs{MPI-ESM LE}. \newline



\begin{figure}[h!]
        \myfloatalign
        \subfloat[\acs{MPI-ESM} advective water transport \text{[Sv]}]
        {\label{fig:UOOC_mean-a}%
       \includegraphics[scale=0.45,page=4,trim=1.4cm 1.7cm 0cm 4.6cm,clip]{UOOC}} \quad
        \subfloat[\acs{MPI-ESM} advective carbon transport \text{[PgC/yr]}]
        {\label{fig:UOOC_mean-b}%
         \includegraphics[scale=0.45,page=1,trim=.8cm 1.7cm 0cm 4.6cm,clip]{UOOC}} \\
        \caption{Zonally averaged transect of the Southern Ocean and the upper-ocean overturning circulation; black arrows show yearly mean advective transports and decadal internal variability $\sigma_{DIV}$ of (a) water in Sv and (b) carbon in PgC/yr; white lines are isopyncals separating \ac{SAMW} at potential density $\rho_{\theta\text{}}=1026.5$ kg m$^{-3}$ from \ac{AAIW}, at $\rho_{\theta\text{}}=1027.2$ kg m$^{-3}$ from \ac{CDW}, at $\rho_{\theta\text{}}=1027.7$ kg m$^{-3}$ from \ac{AABW}; the black line is the \ac{MLD}; the colored contours show the distribution of concentration in \acs{DIC}.} \label{fig:UOOC_mean}
\end{figure}



%\paragraph{Link to the carbon cycle}
How does the upper-ocean overturning circulation relate to the carbon cycle? The concentration of \acs{DIC} in general increases with depth due to the biological pump and remineralization at depth. This is most clearly seen in the subtropics and blurred by processes of mixing at higher latitudes (\autoref{fig:UOOC_mean}). Still, upwelled waters from the deep oceans have a high pCO$_2$ potential at which they would equilibrate when lifted to the surface, so the upwelling super-saturated waters in high latitude waters drives CO$_2$ outgassing and downwelling takes CO$_2$ equilibrated waters into the deeper ocean (\autoref{fig:UOOC_mean-b}) \citep{Morrison2015}.\newline 

%\paragraph{Temporal evolution of the mean state and internal variability} I would need to construct an index

%\paragraph{Trends of extreme carbon trend members}
During the positive CO$_2$ flux trend upper-ocean overturning circulation intensifies. This means enhanced upwelling, which weakens the carbon sink (\autoref{fig:UOOC_neg}) and vice versa strengthens the carbon sink for weaker upper-ocean overturning circulation (\autoref{fig:UOOC_pos}). %[I could to construct an index to have a similar figure as before, but dont know whether what would be really nessessary for the story.]

%\paragraph{Temporal comparison to observational data}  no data; a bit compared to DeVries


%\paragraph{MPI-OM Southern Ocean performance evaluation}  
The global performance of \acs{MPIOM} is discussed in detail in \cite{Jungclaus2013}. The Southern Ocean \acs{SST} warm bias is attributed to an overestimation of downward shortwave radiation into the polar regions \citep{Stevens2013} and causes the underestimation of sea-ice coverage. Additionally, the water-column stability is too low which allows open-ocean convection in the Ross and Weddell Sea and deeper than observed winter mixing. %Heat from the relatively warm circumpolar deep waters warms the subsurface waters \citep{Stoessel2015}. 
Additional freshwater input from melting glaciers distributed by icebergs improves the water-column stability, increases sea-ice and prevents open-ocean convection. A similar effect has the coupling to a higher resolved atmosphere due to additional freshwater input from circulation pattern changes \citep{Stoessel2015}. 

%\subparagraph{Water mass circulation}
The model quality of overturning circulation in the Southern Ocean is analyzed by \cite{Sallee2013a}. \acs{MPI-ESM} realistically simulates subtropical water temperature and Sub-Antarctic Mode Water in general. But the overturning cell is much weaker than in other models.

%\subparagraph{ZMLD comparison} 
The \ac{MLD} is an important measure to assess how climate models represent the Southern Ocean. \acs{MLD} in \acs{MPI-ESM} is overestimated compared to observational data from \acs{NOAA} Atlas \citep{Monterey1997} in the locations of the \ac{ACC} and the Ross and Weddell Sea (\autoref{fig:SO_comp_zmld}) as well as in the zonal average (\autoref{fig:UOOC_mean}). In the Ross and Weddell Sea the \acs{MLD} deepens to few kilometers depth via open-ocean convection \citep{Stoessel2015}, which often appears in climate models but rarely seen in observations \citep{Heuze2013,deLavergne2014}. The weak stratification explains why \acs{MPI-ESM} overestimates \acs{MLD}; especially in winter \citep{Sallee2013}.



\begin{figure}[bth]
        \myfloatalign
        \subfloat[\acs{MPI-ESM} \acs{MLD} ensemble mean \text{[m]}]
        {\label{fig:SO_comp_zmld-a}%
       \includegraphics[scale=.66,page=1,trim=1.25cm 13.3cm 12.08cm 6.5cm,clip]{Overview_SO_MPIOM_comparison.pdf}} %\quad
        \subfloat[\acs{NOAA} \acs{MLD} Atlas mean \text{[m]}]
        {\label{fig:SO_comp_zmld-b}%
         \includegraphics[scale=.66,page=1,trim=9.4cm 13.3cm 2.1cm 6.5cm,clip]{Overview_SO_MPIOM_comparison.pdf}} \\
        \caption{Spatial distribution of \acf{MLD} in the Southern Ocean: (a) \acs{MPI-ESM} ensemble mean climatology, (b) \acs{NOAA} Atlas climatology \citep{Monterey1997}.} \label{fig:SO_comp_zmld}
\end{figure}

%************************************************
%\chapter{Trends in CO$_2$ flux} % $\mathbb{ZNR}$
\chapter{Processes corresponding to multi-year trends in sea-air CO$_2$ flux}
%************************************************
\label{ch:trends}

%\section[Winds determine internal variability]{Winds determine internal variability of the Southern Ocean CO$_2$ flux}
%\section[Westerly winds and Southern Ocean CO$_2$ flux variability]{The role of westerly winds on Southern Ocean CO$_2$ flux variability}
\section{The role of westerly winds in Southern Ocean CO$_2$ flux variability}

%\paragraph{on area 50-60S as driver of Southern Ocean internal variability, correlation plots co2flux and wind trends} 
\cite{Thompson2000} and \cite{Hall2002} describe the \acf{SAM} as the most dominant mode for higher-latitude Southern Hemisphere variability. \citeauthor{Lovenduski2007} [{\color{RoyalBlue}2007}] explain the influence of the westerly winds on the Southern Ocean carbon sink. 

To further assess internal variability due to westerly winds in \acs{MPI}-\acs{ESM} this chapter discusses the effect of westerly winds on the thermal, physical and biological controls of the Southern Ocean carbon sink in \acs{HAMOCC}.\newline

I find a correlation on 8-year trends between \acs{SAM}, which describes the position and latitudinal shift of westerly winds, and the CO$_2$ flux in the area of largest decadal internal variability at 50-60$^\circ$S (\autoref{fig:scatter}). Although the CO$_2$ flux formula (\autoref{sec:HAMOCC}) also depends on the wind speed at 10m height, the driver of the changes in \autoref{fig:scatter} is not the piston velocity $k_w$, but in $\Delta \text{pCO}_2$ \citep{Lovenduski2015}. However, the magnitude of short-term variability on the timescale of days to hours depends highly on wind strength variability whereas the direction of CO$_2$ flux is independent of wind speed. This relationship reveals two distinct regimes of wind-driven CO$_2$ flux signals in this area: 

Intensified and southward-shifted winds, associated with an increasing trend in \acs{SAM}, lead to a positive CO$_2$ flux trend. This Southern Ocean carbon sink response has been suggested frequently for the observed and projected trend in \acs{SAM} \citep{LeQuere2007,Lovenduski2007,Lovenduski2008,Hauck2013,landschuetzer2015}. The related process responses corresponding to stronger winds are explained in \autoref{sec:trends_pos}.

In \acs{MPI-ESM} I also find the reverse case of weakening and northward-shifting westerly winds, associated with a negative trend in \acs{SAM}, which lead to a negative CO$_2$ flux or ocean uptake trend. The responses of processes corresponding to weaker winds are explained in \autoref{sec:trends_neg}. However, observations do not reveal a strong multi-year negative \acs{SAM} trend (\autoref{fig:evolution_SAM}). Likewise, westerly winds did not weaken but continued to increase during the negative CO$_2$ flux trend in the 2000s \citep{landschuetzer2015}. On the contrary, depending on the starting year of the trend period, the \acs{SAM} index slightly weakened in the early 2000s \citep{Marshall2003,Lovenduski2015}, and also \cite{DeVries2017} report a decline in upper-ocean overturning circulation which may be connected to strength and position of winds. \newline

The strong trends originate in strong changes of the position and strength of Southern hemisphere westerly winds and effect of those on ocean circulation. Though the parametrized eddies in \acs{MPI-ESM LE} might allow deeper mixing to sustain for multiple years and hence longer than the seasonal timescale at which the eddies would counteract those trends \citep{Thompson2011}. Only a variable definition of isopyncal thickness diffusion could parametrize the expected eddy response from high-resolution simulations \citep{Gent2011}. These enhanced eddy fluxes in a low-resolution model are discussed in \cite{Lovenduski2013}.


\begin{figure}[h!]
\centering
%		\includegraphics[scale=.5,page=1,angle=90,trim=1.3cm 2.3cm 2.3cm 3cm,clip]{Scatter_trends_bands_ensanom_co2flux_vs_SAM_n1800_1980_1997_trend8_50-59S}
		\captionsetup[subfigure]{labelformat=empty,justification=centering}
		\subfloat[Sea-air CO$_2$ vs. \acf{SAM}]{		
		\includegraphics[scale=.6,page=1,angle=90,trim=1.3cm 2.3cm 3.8cm 4cm,clip]{EGU_new_SAM_Scatter_trends_bands_ensanom_co2flux_vs_SAM_n1800_1980_1997_trend8}}		
		%\includegraphics[scale=.38,page=1,angle=90,trim=1.3cm 2.3cm 2.3cm 3cm,clip]{Scatter_trends_bands_ensanom_co2flux_vs_SAM_n1800_1980_1997_trend8_40-49S}
		\vspace{-2mm}
		\caption{Linear trends in the \ac{SAM} as indicator of wind strength vs. sea-air CO$_2$ flux at 50-60$^\circ$S; each data point represents 8-year trends of a single realization normalized for the ensemble mean trend  between 1980 and 2004; the blue dot represents the most negative monotonic CO$_2$ flux trend; the red dot the positive monotonic CO$_2$ flux trend.}
		\label{fig:scatter}
\end{figure}

To qualitatively understand the mechanisms of the Southern Ocean carbon sink I analyze the drivers of CO$_2$ flux on a process level. This chapter covers qualitative CO$_2$ flux changes with respect to the thermal effect, physical circulation and biology. A quantitative analysis of the different drivers for different regions follows in \autoref{ch:pCO2separation}.

The analysis presented here involves 8-year trends for reasons stated in \autoref{sec:choicetrend} but the results of this chapter apply to various multi-year and decadal trends subject to internal variability (not shown).

The spatial trend patterns of different trend periods mostly appear zonally symmetric, therefore the analysis is not separated into the different Southern Ocean sectors, \eg in the Pacific, Indian or Atlantic sector. Also, the atmospheric circulation change in \acs{MPI-ESM} is too symmetric compared to observations \citep{Haumann2014}. Therefore, the description here is carried out in zonal latitudinal bands keeping the unsymmetrical Southern Ocean dynamics in mind \citep{Sallee2010,Talley2013}. Also due to the underestimated Antarctic sea-ice and open-ocean convection in the Ross and Weddell Sea, I restrict my analysis to the Southern Ocean north of 60$^\circ$S, but the decadal CO$_2$ flux trends have a weak signal south of 60$^\circ$S anyway. 



\clearpage

\section{Positive CO$_2$ flux trends}
\label{sec:trends_pos}

Strong positive CO$_2$ flux trends correlate with stronger westerly winds (\autoref{fig:scatter}) \citep{Lovenduski2007}. The difference $\Delta$pCO$_2$ between oceanic pCO$_{2,\text{ocean}}$ and atmospheric partial pressures pCO$_{2,\text{atm}}$ depicts a cleaner signal than CO$_2$ flux (\autoref{fig:pos-co2flux},\ref{fig:pos-dpco2}) and is independent of wind speed and solubility \citep{Lovenduski2015}. pCO$_{2,\text{ocean}}$ rises stronger than pCO$_{2,\text{atm}}$, so CO$_2$ must be driven by changes in the ocean dynamics (\autoref{fig:pos-dpco2}).
The strongest positive signal occurs in the upwelling at 50-60$^\circ$S, a weaker and more patchy signal occurs in the subduction areas north of 50$^\circ$S, whereas changes in most other areas of the Southern Ocean are insignificant (\autoref{fig:pos-dpco2}). 

Westerly winds decrease at 40-50$^\circ$S and increase at 50-60$^\circ$S, which results in a southward shift of westerlies (\autoref{fig:pos-slp}) represented by the positive trend in \acs{SAM} (\autoref{fig:evolution_SAM}). \newline


The response of the thermal effect, upper-ocean overturning circulation and biology are described in \autoref{sec:trends_pos_thermal}, \ref{sec:trends_pos_circulation} and \ref{sec:trends_pos_biology}, respectively.

\begin{figure}[bth]
        \myfloatalign
        \subfloat[CO$_2$ flux trend \text{[kgC m$^{-2}$yr$^{-1}$ /8yrs]}]
        {\label{fig:pos-co2flux}%
        \includegraphics[scale=1.55,trim=13.2cm 18.75cm 4.5cm 6.3cm,clip]{\memberpositive _positive_trend_8_obgc_overview_co2.pdf}} %\quad
        \subfloat[$\Delta$pCO$_2$ trend \text{[ppm/8yrs]}]
        {\label{fig:pos-dpco2}%
         \includegraphics[scale=1.55,trim=13.2cm 15.9cm 4.5cm 9.225cm,clip]{\memberpositive _positive_trend_8_obgc_overview_co2.pdf}} \\
         
         \subfloat[\ac{SLP} and wind trends \text{[hPa/8yrs]}]
        {\label{fig:pos-slp}%
        \includegraphics[scale=1.55,trim=13.2cm 10.2cm 4.5cm 14.9cm,clip]{\memberpositive _positive_trend_8_obgc_overview_co2.pdf}} %\quad
        \subfloat[\ac{SST} trend \text{[$^{\circ}$C/8yrs]}]
        {\label{fig:pos-sst}%
         \includegraphics[scale=1.55,trim=13.18cm 13.05cm 4.5cm 12.075cm,clip]{\memberpositive _positive_trend_8_obgc_overview_co2.pdf}} \\
        \caption{Linear trends during the most positive monotonic 8-year sea-air CO$_2$ flux trend: (a) sea-air CO$_2$ flux, (b) $\Delta$pCO$_2$, (c) \ac{SLP} and wind vectors overlain as arrows and (d) \acf{SST}; hatched areas indicate where trends are outside the 5\% significance level.} \label{fig:pos}
\end{figure}



\clearpage

\subsection{Changes in the thermal effect during positive CO$_2$ flux trends}
\label{sec:trends_pos_thermal}
The solubility of pCO$_{\text{2,ocean}}$ is primarily sensitive to \acf{SST}. Warmer oceans such as the tropical oceans own a lower solubility than cooler high-latitude oceans - a process referred to as the solubility pump of carbon \citep{VolkHoffert1985}. Likewise, CO$_2$-equilibrated waters outgas when warmed and take up CO$_2$ when cooled.

The difference between the partial pressure of CO$_2$ in the ocean (pCO$_{2,\text{ocean}}$) and the atmosphere (pCO$_{2,\text{atm}}$) is the main changing quantity in the CO$_2$ flux formula \citep{Lovenduski2015} (\autoref{eq:HAMOCC}). The separation by \citeauthor{Takahashi1993} [{\color{RoyalBlue}1993}, {\color{RoyalBlue}2002}] gives insights about the direct influence of \ac{SST} (\autoref{sec:takahashi}). The thermal pCO$_2$ trend is driven by changes in \acs{SST} (\autoref{fig:thermal_pos-a}), whereas the non-thermal $\Delta $pCO$_2$ trend approximates all other changes in pCO$_{2,\text{atm}}$, biology, alkalinity, assuming constant \acs{SST} (\autoref{fig:thermal_pos-b}). The thermal pCO$_2$ trend and non-thermal $\Delta$pCO$_2$ trend approximately add up to the trends in pCO$_2$ \citep{landschuetzer2015}. 
The thermal trend follows the \acs{SST} cooling trend (\autoref{fig:pos-sst}) south of 50$^\circ$S towards negative CO$_2$ flux trends, whereas the warming north of 50$^\circ$S favors outgassing. Increased Ekman transport causes this heat divergence in polar regions and a heat convergence at lower latitudes \citep{Hall2002} (\autoref{fig:UOOC_pos}). The non-thermal component strongly increases south of 50$^\circ$S, so overall the pCO$_{2,\text{ocean}}$ increases faster than pCO$_{2,\text{atm}}$ leading to a positive sea-air CO$_2$ flux trend. %This reflects the enhanced outgassing from increased upwelling. 
The non-thermal and thermal trends combined nearly compensate north of 50$^\circ$S but at 50-60$^\circ$S the outgassing dominates (\autoref{fig:pos-dpco2}). The homogenous increase in atmospheric pCO$_{\text{2,atm}}$ accounts for a -12ppm/8yrs in $\Delta$pCO$_2$ (\autoref{fig:pos-dpco2}). %
These changes are stronger in the summer season, especially the non-thermal component because of the stronger summer \acs{SLP} trend (\autoref{fig:pos_summer}, \ref{fig:pos_winter}). 

\begin{figure}[bth]
        \myfloatalign
        \subfloat[pCO$_{\text{2,thermal}}$ trend \text{[ppm/8yrs]}]
        {\label{fig:thermal_pos-a}%
        \includegraphics[scale=1.55,trim=13.2cm 7.3cm 4.5cm 17.785cm,clip]{\memberpositive _positive_trend_8_obgc_overview_co2.pdf}} %\quad
        \subfloat[$\Delta$pCO$_{\text{2,non-thermal}}$ trend \text{[ppm/8yrs]}]
        {\label{fig:thermal_pos-b}%
         \includegraphics[scale=1.55,trim=13.2cm 4.5cm 4.4cm 20.65cm,clip]{\memberpositive _positive_trend_8_obgc_overview_co2.pdf}} \\
        \caption{Linear trends during the most positive monotonic 8-year sea-air CO$_2$ flux trend: (a) pCO$_{2,\text{thermal}}$ and (b) $\Delta$pCO$_{2,\text{non-thermal}}$; hatched areas indicate where trends are outside the 5\% significance level.} \label{fig:thermal_pos}
\end{figure}


\clearpage

\subsection{Changes in ocean circulation during positive CO$_2$ flux trends}
\label{sec:trends_pos_circulation}

Stronger westerly winds intensify the upper-ocean overturning circulation \citep{Lauderdale2013}. Therefore the circulation field advects more \acs{DIC} and alkalinity along its overturning pathway (\autoref{fig:UOOC_pos}). Intensified upwelling of super-saturated waters at 50-60$^\circ$S increase \acs{DIC} and alkalinity concentrations in the euphotic zone (\autoref{fig:ekman_pos-b}). As the pCO$_{\text{2,ocean}}$ sensitivity of \acs{DIC} is larger than for alkalinity, this likely enhances pCO$_{\text{2,ocean}}$, which leads to a positive CO$_2$ flux (for a detailed dicussion see \autoref{ch:pCO2separation}). The stronger winds also increase Ekman northward transport and advects \acs{DIC} and alkalinity further northward (\autoref{fig:ekman_pos-a}). Subduction rates of \ac{AAIW} and \ac{SAMW} formation increase north of 50$^\circ$S, so pCO$_{2,\text{atm}}$-equilibrated waters take additional anthropogenic carbon into the deeper ocean. The southward shift of westerly winds weakens the northern edge of Ekman transport at 30-40$^\circ$S. 

The changes in \ac{MLD} also contribute to the vertical transport of carbon. By deeper mixing in winter, more carbon-rich waters are included into the \acs{MLD}, which then serves as a larger super-saturated reservoir. \acs{MLD} deepens south of 50$^\circ$S and slightly shoals north of 45$^\circ$S, whereas open-ocean convection causes the unrealistic zonal \acs{MLD} averages below 300m south of 60$^\circ$S (\autoref{fig:UOOC_pos}) \citep{Sallee2013,Stoessel2015}. 



\begin{figure}[bth]
        \myfloatalign
        \subfloat[Ekman transport trend \text{[m$^2$ s$^{-1}$/8yrs]}]
        {\label{fig:ekman_pos-a}%
        \includegraphics[scale=1.5,trim=13.1cm 18.75cm 4.5cm 6.38cm,clip]{\memberpositive _positive_trend_8_ekman_overview.pdf}} %\quad
        \subfloat[Ekman pumping trend \text{[m s$^{-1}$/8yrs]}]
        {\label{fig:ekman_pos-b}%
         \includegraphics[scale=1.5,trim=13.25cm 15.9cm 4.1cm 9.2cm,clip]{\memberpositive _positive_trend_8_ekman_overview.pdf}} \\
        \caption{Linear trends during the most positive 8-year sea-air CO$_2$ flux trend: (a) Ekman transport and (b) Ekman pumping; hatched areas indicate where trends are outside the 5\% significance level.} \label{fig:ekman_pos}
\end{figure}


\begin{figure}[hbt]
	\centering
    \captionsetup[subfigure]{labelformat=empty,justification=centering}
	\subfloat[Carbon advection trends \text{[PgC/8yrs]}]{ 
	\includegraphics[scale=0.42,page=2,trim=.3cm 1.7cm 0cm 4.6cm,clip]{UOOC}}
	%\vspace{-5mm}
	\caption{Zonally averaged upper-ocean overturning circulation during the most positive 8-year sea-air CO$_2$ flux trend; black arrows show mean advective carbon transport, red arrows show advective carbon transport trends enforcing the upper-ocean overturning circulation; blue arrows show advective carbon transport trends weakening the upper-ocean overturning circulation; black numbers quantify the trends in advective carbon transport in PgC/8yrs; white lines are isopyncals as in \autoref{fig:UOOC_mean}; grey line is \ac{MLD} in the beginning and the black line MLD in the end of the period.}
	\label{fig:UOOC_pos}
\end{figure}

The upper-ocean overturning circulation response presented here is in-line with idealized wind-change studies \citep{Lauderdale2013}, modeling studies explaining the observed CO$_2$ flux trend in the 1990s \citep{LeQuere2007,Lovenduski2007,Lovenduski2008} as well an inverse modeling study for the 1990s \citep{DeVries2017}.


\clearpage

\subsection{Changes in biology during positive CO$_2$ flux trends}
\label{sec:trends_pos_biology}
The biological pump draws down surface \acf{DIC}, slighty increases alkalinity and is sensitive to changes in circulation (\autoref{fig:deffeyes_processes}). Mixing changes nutrient distributions, when remineralized nutrients from the deep ocean flush into the euphotic zone. Mixing also pulls the standing stock of phytoplankton deeper into the ocean, where less light inhibits growth. Changes in \acs{SST} also directly affect phytoplankton growth (\autoref{sec:HAMOCC}). 

In this subsection, I analyze the 8-year summer (SONDJF) austral summer trends to understand the trends in primary production.\newline
  
Primary production and sea-air CO$_2$ flux show opposing zonally symmetric trend patterns as phytoplankton growth takes up large amounts of surface \acs{DIC}, but only weakly increases alkalinity  and hence lowers pCO$_2$ (\autoref{fig:pos-co2flux}, \ref{fig:summer_trends_pos-intpp}). Primary production declines most pronounced at 50-60$^\circ$S, increases at 40-50$^\circ$S and declines at 30-40$^\circ$S.  The challenge at hand is to find out which changes drive these different responses.
%\paragraph{on intpp not increasing because of more nutrients} different than \citep{Lovenduski2005} 


Internally varying processes change the availability of nutrients. 
The decline in nutrients in the subtropics at 30-40$^\circ$S reduces primary production, but the nutrient availability factor (\autoref{sec:HAMOCC}) slightly increases south of 50$^\circ$S (\autoref{fig:summer_trends_pos-nutrient}).
Previous observational and modeling studies suggest an increase in primary production, because upwelling brings nutrients, especially iron, from the deep-ocean to the iron-limited surface waters \citep{Lovenduski2005,Hauck2013,wang2012,Tagliabue2014}. Additional iron fosters primary production following the iron-hypothesis \citep{Martin1990Nature,Martin1990}, but observational data for iron is still rather sparse to test this for the whole Southern Ocean \citep{Tagliabue2014}. Contrasting \acs{HAMOCC}, many models reproduce this suggested iron-limitation in the Southern Ocean and hence respond with increasing primary production \citep{wang2012,Hauck2013}.  \newline

If the reduction in primary production at 50-60$^\circ$S cannot be explained by changes in nutrients, what else effects primary production blooms?

The combined light \& temperature limitation is primarily driven by temperature after the insulation excels a threshold in cold waters (\autoref{fig:lighttemplimf}). The strong \acs{SST} cooling trend dominates the light \& temperature limitation of primary production. The strong signal in coastal areas as well as Weddell and Ross Sea is attributed to sea-ice changes and open-ocean convection but has minor effects on the primary production and CO$_2$ flux (\autoref{fig:summer_trends_pos-lighttemplim}). 

This northward Ekman transport could also advect phytoplankton northwards to cause the increase in primary production at 40-50$^\circ$S (\autoref{fig:ekman_pos-b}, \ref{fig:summer_trends_pos-intpp}).

The overall decline of primary production in the Southern Ocean under a positive \acs{SAM} trend is also related to mixing. The summer \ac{MLD} has a strong increasing trend at 50-60$^\circ$S, so the mixing deepens (\autoref{fig:summer_trends_pos-zmld}). This is caused by stronger winds (\autoref{fig:pos-slp}) and shown in the average depth of the vertical diffusivity due to wind (\autoref{fig:summer_trends_pos-windmixing}). This deeper mixing in summer then mixes the standing stock of phytoplankton to deeper levels, where they are exposed to less light (\autoref{fig:summer_trends_pos-phydepth}) \citep{Margalef1997}. This theory of a critical depth for phytoplankton blooms was initially proposed by \cite{Sverdrup1953} and requires a stable water column for phytoplankton to initiate blooms. However, this theory is based on turbulent mixing, so the oceanographic \acs{MLD} only serves as a first-order mixing measure for phytoplankton \citep{Franks2014}. Still the signal sustains to phytoplankton depth, where the average phytoplankton depth decreases up to 15m resulting in up to 30\% less light. The lack of monthly output for 3D biogeochemical variables made me use annual averages for phytoplankton, which causes the low significance in \autoref{fig:summer_trends_pos-phydepth}. 

The reverse processes contribute to the increase at 40-50$^\circ$S. Less winds mix less deep and allow phytoplankton to stay more confined to the surface, where they get more light and flourish (\autoref{fig:pos-slp}, \ref{fig:summer_trends_pos-windmixing}, \ref{fig:summer_trends_pos-phydepth}). Also the warming increases the phytoplankton growth rate (\autoref{fig:summer_trends_pos-lighttemplim}).\newline

Summarizing, a multitude of interconnected processes cause the decline in primary production in the Southern Ocean for an increasing \acs{SAM} trend. A clear separation of the magnitude of these interdependent processes is impossible.
 
\begin{figure}[bth]
        \myfloatalign
        \subfloat[INTPP trend \text{[kgCm$^{-2}$yr$^{-1}$/8yrs]}]
        {\label{fig:summer_trends_pos-intpp}%
        \includegraphics[scale=1.55,trim=13.2cm 15.9cm 4.4cm 9.225cm,clip]{\memberpositive _positive_trend_8_obgc_overview_summer.pdf}} %\quad
        \subfloat[Nutrient availability trend \text{[\%/8yrs]}]
        {\label{fig:summer_trends_pos-nutrient}%
         \includegraphics[scale=1.55,trim=13.2cm 13.05cm 4.4cm 12.075cm,clip]{\memberpositive _positive_trend_8_obgc_overview_summer.pdf}} \\
         
         \subfloat[Light- \& temp. limitation trend \text{[\%/8yrs]}]
        {\label{fig:summer_trends_pos-lighttemplim}%
       \includegraphics[scale=1.55,trim=13.2cm 4.5cm 4.4cm 20.65cm,clip]{\memberpositive _positive_trend_8_obgc_overview_summer.pdf}} %\quad
        \subfloat[\acf{MLD} \text{[m/8yrs]}]
        {\label{fig:summer_trends_pos-zmld}%
        \includegraphics[scale=1.55,trim=13.2cm 7.3cm 4.4cm 17.775cm,clip]{\memberpositive _positive_trend_8_obgc_overview_summer.pdf}} \\
         
         \subfloat[Wind mixing depth trend \text{[m/8yrs]}]
        {\label{fig:summer_trends_pos-windmixing}%
        \includegraphics[scale=1.55,trim=13.2cm 15.9cm 4.4cm 9.2cm,clip]{\memberpositive _positive_trend_8_schwerpunkt_mixing_overview.pdf}} %\quad
        \subfloat[Phytoplankton depth trend \text{[m/8yrs]}]
        {\label{fig:summer_trends_pos-phydepth}%
         \includegraphics[scale=1.55,trim=13.2cm 18.7cm 4.4cm 6.38cm,clip]{\memberpositive _positive_trend_8_schwerpunkt_mixing_overview.pdf}} \\
        \caption{Southern Ocean austral summer trends during the most positive 8-year sea-air CO$_2$ flux trend: (a) \acf{INTPP}, (b) nutrient availability factor, (c) surface temperature \& limitation function, (d) \acf{MLD}, (e) average depth of vertical diffusivity due to wind and (f) phytoplankton average depth; hatched areas indicate where trends are outside the 5\% significance level.} \label{fig:summer_trends_pos}
\end{figure}



\clearpage



\section{Negative CO$_2$ flux trends}
\label{sec:trends_neg}

Disclaimer: Generally, in this analysis the direction of trends reverse for weakening and northward-shifting westerlies, which leads to an overall negative sea-air CO$_2$ flux trend. Additionally, now the prescribed atmospheric pCO$_{\text{2,atm}}$ forcing promotes a steady negative background sea-air CO$_2$ flux trend.\newline

A strong negative sea-air CO$_2$ flux trends correlates with weaker westerly winds (\autoref{fig:neg-co2flux}, \ref{fig:neg-slp}). The strongest negative signal $\Delta$pCO$_2$ occurs in the upwelling area at 50-60$^\circ$S. Changes in most other areas of the Southern Ocean are insignificant (\autoref{fig:neg-dpco2}). 

%Westerly winds decrease at 40-60$^\circ$S, which results in a northward shift of westerlies (\autoref{fig:pco2_pos}) represented by the negative trend in \acs{SAM} (\autoref{fig:evolution_SAM}).

%The strengthening of the Southern Ocean carbon sink under the context of intensified westerly winds leads me to a hypothesis (fig. \autoref{fig:schematics_pos}):
%Weaker winds at 50-60$^\circ$S slow down the upper-ocean overturning circulation and increase primary production due to a more stable water column. [really unsure about this hypothesis part - I dont discover anything new but want to explain the process with a figure ahead]
The response in the thermal effect, upper-ocean overturning circulation and biology are described in \autoref{sec:trends_neg_thermal}, \ref{sec:trends_neg_circulation} and \ref{sec:trends_neg_biology}, respectively.

\begin{figure}[bth]
        \myfloatalign
        \subfloat[CO$_2$ flux trend \text{[kgC m$^{-2}$yr$^{-1}$ /8yrs]}]
        {\label{fig:neg-co2flux}%
        \includegraphics[scale=1.55,trim=13.2cm 18.75cm 4.5cm 6.3cm,clip]{\membernegative _positive_trend_8_obgc_overview_co2.pdf}} %\quad
        \subfloat[$\Delta$pCO$_2$ trend \text{[ppm/8yrs]}]
        {\label{fig:neg-dpco2}%
         \includegraphics[scale=1.55,trim=13.2cm 15.9cm 4.5cm 9.225cm,clip]{\membernegative _positive_trend_8_obgc_overview_co2.pdf}} \\
         
         \subfloat[\ac{SLP} and wind trends \text{[hPa/8yrs]}]
        {\label{fig:neg-slp}%
        \includegraphics[scale=1.55,trim=13.2cm 10.2cm 4.5cm 14.9cm,clip]{\membernegative _positive_trend_8_obgc_overview_co2.pdf}} %\quad
        \subfloat[\ac{SST} trend \text{[$^{\circ}$C/8yrs]}]
        {\label{fig:neg-sst}%
         \includegraphics[scale=1.55,trim=13.18cm 13.05cm 4.5cm 12.075cm,clip]{\membernegative _positive_trend_8_obgc_overview_co2.pdf}} \\
        \caption{Linear trends during the most negative monotonic 8-year air-sea CO$_2$ flux trend: (a) sea-air CO$_2$ flux, (b) $\Delta$pCO$_2$, (c) \ac{SLP} and wind vectors overlain as arrows and (d) \acf{SST}; hatched areas indicate where trends are outside the 5\% significance level.} \label{fig:neg}
\end{figure}



\clearpage

\subsection{Changes in the thermal effect during negative CO$_2$ flux trends}
\label{sec:trends_neg_thermal}

The \ac{SST} warming trend (\autoref{fig:neg-sst}) drives the thermal trend south of 50$^\circ$S towards a positive pCO$_{2,\text{ocean}}$ trend, whereas the cooling north of 50$^\circ$S results in a pCO$_{2,\text{ocean}}$ decline (\autoref{fig:thermal_neg-a}). These \acs{SST} changes are caused by less Ekman transport and lead to a heat convergence in the polar region and a heat divergence in the subtropics \citep{Hall2002}. The non-thermal trend has a strong negative signal south of 50$^\circ$S and a slightly positive north of 50$^\circ$S (\autoref{fig:thermal_neg-b}). The increase in atmospheric pCO$_{\text{2,atm}}$ accounts for a -14ppm/8yrs. The non-thermal and thermal trends combined nearly compensates north of 50$^\circ$S, but the non-thermal component dominates at 50-60$^\circ$S (\autoref{fig:neg-dpco2}). 

These changes are stronger in the summer season, especially the non-thermal component, because of the stronger \acs{SLP} winter trends (\autoref{fig:neg_summer}, \ref{fig:neg_winter}). 
 
\begin{figure}[bth]
        \myfloatalign
        \subfloat[pCO$_{\text{2,thermal}}$ trend \text{[ppm/8yrs]}]
        {\label{fig:thermal_neg-a}%
        \includegraphics[scale=1.55,trim=13.2cm 7.3cm 4.5cm 17.785cm,clip]{\membernegative _positive_trend_8_obgc_overview_co2.pdf}} %\quad
        \subfloat[$\Delta$pCO$_{\text{2,non-thermal}}$ trend \text{[ppm/8yrs]}]
        {\label{fig:thermal_neg-b}%
         \includegraphics[scale=1.55,trim=13.2cm 4.5cm 4.4cm 20.65cm,clip]{\membernegative _positive_trend_8_obgc_overview_co2.pdf}} \\
        \caption{Linear trends during the most negative monotonic 8-year sea-air CO$_2$ flux trend: (a) pCO$_{2,\text{thermal}}$ and (b) $\Delta$pCO$_{2,\text{non-thermal}}$; hatched areas indicate where trends are outside the 5\% significance level.} \label{fig:thermal_neg}
\end{figure}




\clearpage

\subsection{Changes in ocean circulation during negative CO$_2$ flux trends}
\label{sec:trends_neg_circulation}

Weaker westerly winds weaken the upper-ocean overturning circulation \citep{Lauderdale2013}, so the circulation field advects less \acs{DIC} and alkalinity along its overturning pathway (\autoref{fig:UOOC_neg}). Less upwelling at 50-60$^\circ$S decreases \acs{DIC} and alkalinity concentrations in the euphotic zone (\autoref{fig:ekman_neg-pumping}). Weaker winds also decrease Ekman northward transport and reduce northward advection of \acs{DIC} and alkalinity (\autoref{fig:ekman_neg-transport}). North of 50$^\circ$S subduction rates of \acs{AAIW} and \acs{SAMW} formation decrease, which could also be a sign of a northward-shift in upwelling. The \acs{MLD} shoals south of 50$^\circ$S and slightly deepens north of 45$^\circ$S (\autoref{fig:UOOC_neg}). 


\begin{figure}[hbt]
        \myfloatalign
        \subfloat[Ekman transport trend \text{[m$^2$ s$^{-1}$/8yrs]}]
        {\label{fig:ekman_neg-transport}%
        \includegraphics[scale=1.5,trim=13.1cm 18.75cm 4.5cm 6.38cm,clip]{\membernegative _positive_trend_8_ekman_overview.pdf}} %\quad
        \subfloat[Ekman pumping trend \text{[m s$^{-1}$/8yrs]}]
        {\label{fig:ekman_neg-pumping}%
         \includegraphics[scale=1.5,trim=13.25cm 15.9cm 4.1cm 9.2cm,clip]{\membernegative _positive_trend_8_ekman_overview.pdf}} \\
        \caption{Linear trends during the most negative 8-years sea-air CO$_2$ flux trend: (a) Ekman transport and (b) Ekman pumping; hatched areas indicate where trends are outside the 5\% significance level.} \label{fig:ekman_neg}
\end{figure}


\begin{figure}[hbt]
	\centering
	\captionsetup[subfigure]{labelformat=empty,justification=centering}
	\subfloat[Carbon advection trends \text{[PgC/8yrs]}]{ 
	\includegraphics[scale=0.42,page=3,trim=.3cm 1.7cm 0cm 4.cm,clip]{UOOC}}
	%\includegraphics[scale=0.5,page=3,trim=1.4cm 1.7cm 0cm 4.6cm,clip]{UOOC}
	%\vspace{-10mm}
	\caption{Zonally averaged upper-ocean overturning circulation during the most negative 8-year sea-air CO$_2$ flux trend; black arrows show mean advective carbon transport, red arrows show advective carbon transport trends enforcing the upper-ocean overturning circulation; blue arrows show advective carbon transport trends weakening the upper-ocean overturning circulation; black numbers show the trends in advective carbon transport in PgC/8yrs; white lines are isopyncals as in \autoref{fig:UOOC_mean}; grey line is \ac{MLD} in the beginning and the black line MLD in the end of the period.}%; black arrows show mean advective carbon transport, red arrows show advective carbon transport trend over 8 years; white lines are isopyncals separating Sub-Antarctic Mode Water (SAMW), Antarctic Intermediate Water (AAIW), Circumpolar Deep Water (CDW) and Antarctic Bottom Water (AABW)}
	\label{fig:UOOC_neg}
\end{figure}


This upper-ocean overturning circulation response agrees with idealized wind-change studies \citep{Lauderdale2013}. The inverse modeling study reports a decline from the 2000s towards the previous decade \citep{DeVries2017}, which could be related to changes in wind. However, \cite{landschuetzer2015} did not report a strong negative trend for the strength of observed westerly winds, but a pattern change in \acs{SLP} in the 2000s became more zonally asymmetric. On the contrary, depending on the starting year of the trend period, the \acs{SAM} index slightly weakened in the early 2000s \citep{Marshall2003,Lovenduski2015}.


\clearpage

\subsection{Changes in biology during negative CO$_2$ flux trends}
\label{sec:trends_neg_biology}

%The biological pump draws down surface DIC and sensitive to changes in circulation (see section \autoref{sec:HAMOCC}). In this subsection, I analyze the 8-year summer (SONDJF) austral summer trends. 
  
Primary production and CO$_2$ flux show opposing zonally symmetric trend patterns (\autoref{fig:neg-co2flux}, \ref{fig:summer_trends_neg-intpp}). The patchy trend patterns of primary production increase  most dominant at 50-60$^\circ$S, decrease at 40-50$^\circ$S and increase at 30-40$^\circ$S. %Which changes drive these different responses?
%\paragraph{on intpp not increasing because of more nutrients} different than \citep{Lovenduski2005} 

%Internally varying processes change the availability of nutrients. 
The additional supply of nutrients in the subtropics at 30-40$^\circ$S fosters primary production, because the northward-shift in upwelling flushes nutrients from the deep into the nutrient-depleted subtropical gyre region. South of 45$^\circ$S the nutrient availability hardly changes and decreases (\autoref{fig:summer_trends_neg-nutrient}). 
Previous observational and modeling studies suggest decreasing in primary production because less upwelling brings less nutrients, especially iron, from the deep-ocean to the surface \citep{Hauck2013,Tagliabue2014}. But the Southern Ocean in \acs{HAMOCC} is nitrate limited, so the slight reduction in nutrients rather originates in the increased nutrient consumption due to primary production.  %\newline

%If the reduction in primary production at 50-60$^\circ$S cannot be explained by changes in nutrients, what else effects primary production blooms?

The \acs{SST} trends enhance primary production at 50-60$^\circ$S and lower primary production north of 50$^\circ$S (\autoref{fig:summer_trends_neg-lighttemplim}). %The strong light \& temperature limitation signal in coastal areas as well as Weddell and Ross Sea is attributed to sea-ice changes and open-ocean convection, but has minor effects on the primary production and CO$_2$ flux (fig. \autoref{fig:summer_trends_neg}d). 

The weakened northward Ekman transport also keeps the standing stock more southwards and causes the increase in primary production at 50-60$^\circ$S and the shifted decrease at 40-50$^\circ$S (\autoref{fig:ekman_neg}). 

Weaker winds mix the ocean less deep and thereby keeps the standing stock of phytoplankton in more light-flooded levels at 50-60$^\circ$S (\autoref{fig:summer_trends_neg-windmixing}).
The reverse process contributes to the decrease at 40-50$^\circ$S.\newline

Overall, primary production decreases for weaker and northward-shifting winds.
%Summarizing, a multitude of interconnected processes caused the increase in primary production in the Southern Ocean for a decreasing SAM trend. A clear separation of the magnitude of the different effects is impossible.
 


\begin{figure}[bth]
        \myfloatalign
        \subfloat[INTPP trend \text{[kgCm$^{-2}$yr$^{-1}$/8yrs]}]
        {\label{fig:summer_trends_neg-intpp}%
        \includegraphics[scale=1.55,trim=13.2cm 15.9cm 4.4cm 9.225cm,clip]{\membernegative _positive_trend_8_obgc_overview_summer.pdf}} %\quad
        \subfloat[Nutrient availability trend \text{[\%/8yrs]}]
        {\label{fig:summer_trends_neg-nutrient}%
         \includegraphics[scale=1.55,trim=13.2cm 13.05cm 4.4cm 12.075cm,clip]{\membernegative _positive_trend_8_obgc_overview_summer.pdf}} \\
         
         \subfloat[Light- \& temp. limitation trend \text{[\%/8yrs]}]
        {\label{fig:summer_trends_neg-lighttemplim}%
       \includegraphics[scale=1.55,trim=13.2cm 4.5cm 4.4cm 20.65cm,clip]{\membernegative _positive_trend_8_obgc_overview_summer.pdf}} %\quad
        \subfloat[\acf{MLD} \text{[m/8yrs]}]
        {\label{fig:summer_trends_neg-zmld}%
        \includegraphics[scale=1.55,trim=13.2cm 7.3cm 4.4cm 17.775cm,clip]{\membernegative _positive_trend_8_obgc_overview_summer.pdf}} \\
         
         \subfloat[Wind mixing depth trend \text{[m/8yrs]}]
        {\label{fig:summer_trends_neg-windmixing}%
        \includegraphics[scale=1.55,trim=13.2cm 15.9cm 4.4cm 9.2cm,clip]{\membernegative _positive_trend_8_schwerpunkt_mixing_overview.pdf}} %\quad
        \subfloat[Phytoplankton depth trend \text{[m/8yrs]}]
        {\label{fig:summer_trends_neg-phydepth}%
         \includegraphics[scale=1.55,trim=13.2cm 18.7cm 4.4cm 6.38cm,clip]{\membernegative _positive_trend_8_schwerpunkt_mixing_overview.pdf}} \\
        \caption{Southern Ocean austral summer trends during the most negative sea-air 8-year CO$_2$ flux trend: (a) vertically integrated primary production (INTPP), (b) nutrient availability factor, (c) surface temperature \& limitation function, (d) \acf{MLD}, (e) average depth of vertical diffusivity due to wind and (f) phytoplankton average depth; hatched areas indicate where trends are outside the 5\% significance level.} \label{fig:summer_trends_neg}
\end{figure}
%************************************************
\chapter{Discussion}\label{ch:discussion} % $\mathbb{ZNR}$
%************************************************

%\section{Some Formulas}
\citep{landschuetzer2015} \\
\citep{LeQuere2007}\\
\citep{Lovenduski2005}\\ 
\citep{Lovenduski2007}\\
\citep{Lovenduski2008}\\
\citep{Hauck2013}\\
\citep{wang2012}\\
\vspace{3cm}

\citep{Lovenduski2005}: how is model different, how are results different
\begin{enumerate}
\item our MLD responds differently than Lovenduski2005 Fig.2 schematic illustration
\item increase in SAM --> increase in chlorophyll south of PF/45°S because of iron supply from below (we have phy-decrease)
\item increase in SAM --> reduction in chlorophyll north of PF/45°S because of increased ZMLD (we have a slight phy-increase)
\item BUT we dont have iron limitation in SO, so this study can be used as a direct physical effect of stronger winds on primary production excluding nutrient supply changes
\end{enumerate} 

\citep{Hauck2013}: stronger winds, increased upwelling, more iron supply, higher primary production

\citep{wang2012}: 
\begin{enumerate}
\item forced model, 25yr trends, so rather climate change impact than decadal variability
\item some regions: more upwelling, more iron, more primary production 
\item I dont get the clear storyline of the whole paper
\end{enumerate}

Limitations of HAMOCC vs obs/other models:
\begin{enumerate}
\item too early and enhanced Southern Ocean seasonal cycle \citep{Nevison2016}
\item not eddy resolving, has impacts \citep{somestudy}
\item location of polar jets in ECHAM %[I remember this from a group meeting, but didnt check so far]
\item basically no nutrient limitation in SO? - other models and observations disagree
\item other observation studies use sparse for pCO2 or proxy (satelite chl-a)
\item MPIOM performance on circulation patterns and water masses in SO %(Yohei mentioned Anne M. could know this)
\item only one phytoplankton type in HAMOCC? - more variety of plankton species might be able to adapt to depth? or is this too unrealistic for a global model such as HAMOCC?
\item MPI-ESM zmld performance in SO \citep{Sallee2013}
\end{enumerate} 
%************************************************
\chapter{Conclusion and Summary}\label{ch:summary} % $\mathbb{ZNR}$
%************************************************

MPI-ESM large ensemble simulations produces internal variability of the Southern Ocean carbon sink including decreasing decadal carbon sink trends over the past decades, which were seen in observations. Decreasing carbon sink trends are accompanied by intensified winds as in previous studies \citep{LeQuere2007,Lovenduski2008}. The changes in circulation induce a different response to biology as in previous studies\citep{Lovenduski2005,Hauck2013,wang2012}, which is mainly attributed to differences in nutrient availability. Our results suggest that increasing winds not only enhance upwelling of carbon-rich waters, but also decrease water column stability which inhibits primary production. An overall decrease in primary production decreases the Southern Ocean carbon sink. 




%\include{multiToC} % <--- just debug stuff, ignore for your documents
% ********************************************************************
% Backmatter
%*******************************************************
\appendix
%\renewcommand{\thechapter}{\alph{chapter}}
\cleardoublepage
%\part{Appendix}
%********************************************************************
% Appendix
%*******************************************************
% If problems with the headers: get headings in appendix etc. right
%\markboth{\spacedlowsmallcaps{Appendix}}{\spacedlowsmallcaps{Appendix}}
\chapter{Appendix}
\addtocontents{toc}{\protect\setcounter{tocdepth}{0}}

\renewcommand{\thefigure}{A\arabic{figure}}
%\section{Supplementary information}
\setcounter{figure}{0}
%%%%%%%%\counterwithin{figure}{section}
\section{Statistics of Southern Ocean carbon sink}
\begin{figure}[h!]
        \myfloatalign
        \captionsetup[subfigure]{justification=centering}
        \subfloat[Southern Ocean sea-air CO$_2$ flux \text{[PgC/yr]}]
        {\label{fig:SOCS_temporal_gaussian-a}%
        \includegraphics[scale=.53,trim=0cm 0cm 0cm 1cm,clip]{new_SOCS_co2flux_Distribution_n24.png}} \quad
        \subfloat[Southern Ocean sea-air CO$_2$ flux of a single grid cell \text{[kgC m$^{-2}$s$^{-1}$]}]
        {\label{fig:SOCS_temporal_gaussian-b}%
         \includegraphics[scale=.53,trim=0cm 0cm 0cm 1cm,clip]{new_SO_co2flux_Spatial_Distribution_n24.png}} \\
        \caption{Propability distribution function of the annual sea-air CO$_2$ flux in the Southern Ocean in ensemble space and temporal space between 1980-2004: (a) field sum over 35-90$^\circ$S and (b) in a random grid cell.} \label{fig:SOCS_temporal_gaussian}
\end{figure}

\clearpage
\section{CO$_2$ flux trends}
\begin{figure}[h!]
	\topinset{\textbf{*}}{\topinset{\textbf{*}}{\topinset{\textbf{*}}{\topinset{\textbf{*}}{\includegraphics[scale=.75,angle=0,trim=1.4cm 1.3cm 1cm 2cm,clip]{heatmap_not_de-ens-trended.pdf}}{11.1cm}{9.1cm}}{7.75cm}{9.1cm}}{7.75cm}{4.8cm}}{11.1cm}{4.4cm} % from gfx folder
	%\includegraphics[scale=.64,angle=90,trim=1.3cm 1.3cm 5cm 1cm,clip]{heatmap_de-ens-trended.pdf} % from gfx folder
\caption{Southern Ocean carbon sink trends per trend length; * indicate the strongest 8-year and 10-year trends in \ac{SOM-FFN}}
	\label{fig:heatmap}
\end{figure}

\begin{figure}[h!]
	%\includegraphics[scale=.75,angle=0,trim=1.4cm 1.3cm 1cm 1cm,clip]{heatmap_not_de-ens-trended.pdf} % from gfx folder
	\includegraphics[scale=.75,angle=0,trim=1.4cm 1.3cm 1cm 1cm,clip]{heatmap_de-ens-trended.pdf} % from gfx folder
\caption{Southern Ocean carbon sink trends per trend length; corrected for forced trend}
	\label{fig:heatmap_detrended}
\end{figure}



\clearpage
\section{Model evaluation additum}
\begin{figure}[bth]
        \myfloatalign
        \subfloat[\acs{MPI-ESM} surface \acs{DIC} ensemble \text{      }  \text{      } mean \text{[mmol C m$^{-3}$]}]
        {\label{fig:SOCS_comp_DIC-a}%
       \includegraphics[scale=.66,page=2,trim=1.3cm 13.3cm 12.1cm 6.5cm,clip]{Overview_SO_nutrient_comparison.pdf}} %\quad
        \subfloat[\acs{GLODAP}v2 surface \acs{DIC} mean \text{      }  \text{      } \text{      }  \text{      } \text{      }  \text{      }\text{[mmol C m$^{-3}$]}]
        {\label{fig:SOCS_comp_DIC-b}%
         \includegraphics[scale=.66,page=2,trim=9.4cm 13.3cm 2.1cm 6.5cm,clip]{Overview_SO_nutrient_comparison.pdf}} \\
       \caption{Spatial distribution of the climatology of surface \acf{DIC}: (a) \acs{MPI-ESM LE} climatology, (b) \acf{GLODAP} Version 2 climatology \citep{GLODAPv2}.} \label{fig:SOCS_comp_DIC}
\end{figure}

\begin{figure}[bth]
        \myfloatalign
        \subfloat[\acs{MPI-ESM} surface phosphate ensemble \text{      }  \text{      } mean \text{[mmol P m$^{-3}$]}]
        {\label{fig:SOCS_comp_phosph-a}%
       \includegraphics[scale=.66,page=1,trim=1.3cm 13.3cm 12.1cm 6.5cm,clip]{Overview_SO_nutrient_comparison.pdf}} %\quad
        \subfloat[\acs{WOA} surface phosphate mean \text{      }  \text{      } \text{      }  \text{      } \text{      }  \text{      }\text{[mmol P m$^{-3}$]}]
        {\label{fig:SOCS_comp_phosph-b}%
         \includegraphics[scale=.66,page=1,trim=9.4cm 13.3cm 2.1cm 6.5cm,clip]{Overview_SO_nutrient_comparison.pdf}} \\
       \caption{Spatial distribution of the climatology of surface phosphate: (a) \acs{MPI-ESM LE} climatology, (b) \acf{WOA} climatologicy \citep{WOA2013}.} \label{fig:SOCS_comp_phosph}
\end{figure}

\begin{figure}[bth]
        \myfloatalign
        \subfloat[\acs{MPI-ESM} \acs{SST} ensemble mean \text{[$^\circ$C]}]
        {\label{fig:SOCS_comp_SST-a}%
       \includegraphics[scale=.66,page=2,trim=1.3cm 13.3cm 12.1cm 6.5cm,clip]{Overview_SO_MPIOM_comparison.pdf}} %\quad
        \subfloat[\acs{WOA} \acs{SST} mean \text{[$^\circ$C]}]
        {\label{fig:SOCS_comp_SST-b}%
         \includegraphics[scale=.66,page=2,trim=9.4cm 13.3cm 2.1cm 6.5cm,clip]{Overview_SO_MPIOM_comparison.pdf}} \\
       \caption{Spatial distribution of the climatology of \acf{SST}: (a) \acs{MPI-ESM LE} climatology, (b) WOCE climatology \citep{WOCE}.} \label{fig:SOCS_comp_SST}
\end{figure}

\clearpage
\section{Thermal separation by seasons}
\subsection{Positive CO$_2$ flux trend}

\begin{figure}[h!]
        \myfloatalign
        \subfloat[CO$_2$ flux trend \text{[kgC m$^{-2}$yr$^{-1}$ /8yrs]}]
        {\label{fig:pos_summer-co2flux}%
        \includegraphics[scale=1.55,trim=13.2cm 18.75cm 4.5cm 6.3cm,clip]{\memberpositive _positive_trend_8_obgc_overview_co2_summer.pdf}} %\quad
        \subfloat[$\Delta$pCO$_2$ trend \text{[ppm/8yrs]}]
        {\label{fig:pos_summer-dpco2}%
         \includegraphics[scale=1.55,trim=13.2cm 15.9cm 4.5cm 9.225cm,clip]{\memberpositive _positive_trend_8_obgc_overview_co2_summer.pdf}} \\
         
         \subfloat[\ac{SLP} and wind trends \text{[hPa/8yrs]}]
        {\label{fig:pos_summer-slp}%
        \includegraphics[scale=1.55,trim=13.2cm 10.2cm 4.5cm 14.9cm,clip]{\memberpositive _positive_trend_8_obgc_overview_co2_summer.pdf}} %\quad
        \subfloat[\ac{SST} trend \text{[$^{\circ}$C/8yrs]}]
        {\label{fig:pos_summer-sst}%
         \includegraphics[scale=1.55,trim=13.18cm 13.05cm 4.5cm 12.075cm,clip]{\memberpositive _positive_trend_8_obgc_overview_co2_summer.pdf}} \\ 
         
         \subfloat[pCO$_{\text{2,thermal}}$ trend \text{[ppm/8yrs]}]
        {\label{fig:thermal_summer_pos-a}%
        \includegraphics[scale=1.55,trim=13.2cm 7.3cm 4.5cm 17.785cm,clip]{\memberpositive _positive_trend_8_obgc_overview_co2_summer.pdf}} 
        \subfloat[$\Delta$pCO$_{\text{2,non-thermal}}$ trend \text{[ppm/8yrs]}]
        {\label{fig:thermal_summer_pos-b}%
         \includegraphics[scale=1.55,trim=13.2cm 4.5cm 4.4cm 20.65cm,clip]{\memberpositive _positive_trend_8_obgc_overview_co2_summer.pdf}}
        \caption{Linear trends in austral summer during the most positive monotonic 8-year air-sea CO$_2$ flux trend: (a) sea-air CO$_2$ flux, (b) $\Delta$pCO$_2$, (c) \ac{SLP} and wind vectors overlain as arrows, (d) \acf{SST}, (e) pCO$_{2,\text{thermal}}$ and (f) $\Delta$pCO$_{2,\text{non-thermal}}$; hatched areas indicate where trends are outside the 5\% significance level.} \label{fig:pos_summer}
\end{figure}



\begin{figure}[h!]
        \myfloatalign
        \subfloat[CO$_2$ flux trend \text{[kgC m$^{-2}$yr$^{-1}$ /8yrs]}]
        {\label{fig:pos_winter-co2flux}%
        \includegraphics[scale=1.55,trim=13.2cm 18.75cm 4.5cm 6.3cm,clip]{\memberpositive _positive_trend_8_obgc_overview_co2_winter.pdf}} %\quad
        \subfloat[$\Delta$pCO$_2$ trend \text{[ppm/8yrs]}]
        {\label{fig:pos_winter-dpco2}%
         \includegraphics[scale=1.55,trim=13.2cm 15.9cm 4.5cm 9.225cm,clip]{\memberpositive _positive_trend_8_obgc_overview_co2_winter.pdf}} \\
         
         \subfloat[\ac{SLP} and wind trends \text{[hPa/8yrs]}]
        {\label{fig:pos_winter-slp}%
        \includegraphics[scale=1.55,trim=13.2cm 10.2cm 4.5cm 14.9cm,clip]{\memberpositive _positive_trend_8_obgc_overview_co2_winter.pdf}} %\quad
        \subfloat[\ac{SST} trend \text{[$^{\circ}$C/8yrs]}]
        {\label{fig:pos_winter-sst}%
         \includegraphics[scale=1.55,trim=13.18cm 13.05cm 4.5cm 12.075cm,clip]{\memberpositive _positive_trend_8_obgc_overview_co2_winter.pdf}} \\ 
         
         \subfloat[pCO$_{\text{2,thermal}}$ trend \text{[ppm/8yrs]}]
        {\label{fig:thermal_winter_pos-a}%
        \includegraphics[scale=1.55,trim=13.2cm 7.3cm 4.5cm 17.785cm,clip]{\memberpositive _positive_trend_8_obgc_overview_co2_winter.pdf}} 
        \subfloat[$\Delta$pCO$_{\text{2,non-thermal}}$ trend \text{[ppm/8yrs]}]
        {\label{fig:thermal_winter_pos-b}%
         \includegraphics[scale=1.55,trim=13.2cm 4.5cm 4.4cm 20.65cm,clip]{\memberpositive _positive_trend_8_obgc_overview_co2_winter.pdf}}
        \caption{Linear trends in austral winter during the most positive monotonic 8-year sea-air CO$_2$ flux trend: (a) sea-air CO$_2$ flux, (b) $\Delta$pCO$_2$, (c) \ac{SLP} and wind vectors overlain as arrows, (d) \acf{SST}, (e) pCO$_{2,\text{thermal}}$ and (f) $\Delta$pCO$_{2,\text{non-thermal}}$; hatched areas indicate where trends are outside the 5\% significance level.} \label{fig:pos_winter}
\end{figure}


\clearpage
\section{Negative CO$_2$ flux trend}
\begin{figure}[h!]
        \myfloatalign
        \subfloat[CO$_2$ flux trend \text{[kgC m$^{-2}$yr$^{-1}$ /8yrs]}]
        {\label{fig:neg_summer-co2flux}%
        \includegraphics[scale=1.55,trim=13.2cm 18.75cm 4.5cm 6.3cm,clip]{\membernegative _positive_trend_8_obgc_overview_co2_summer.pdf}} %\quad
        \subfloat[$\Delta$pCO$_2$ trend \text{[ppm/8yrs]}]
        {\label{fig:neg_summer-dpco2}%
         \includegraphics[scale=1.55,trim=13.2cm 15.9cm 4.5cm 9.225cm,clip]{\membernegative _positive_trend_8_obgc_overview_co2_summer.pdf}} \\
         
         \subfloat[\ac{SLP} and wind trends \text{[hPa/8yrs]}]
        {\label{fig:neg_summer-slp}%
        \includegraphics[scale=1.55,trim=13.2cm 10.2cm 4.5cm 14.9cm,clip]{\membernegative _positive_trend_8_obgc_overview_co2_summer.pdf}} %\quad
        \subfloat[\ac{SST} trend \text{[$^{\circ}$C/8yrs]}]
        {\label{fig:neg_summer-sst}%
         \includegraphics[scale=1.55,trim=13.18cm 13.05cm 4.5cm 12.075cm,clip]{\membernegative _positive_trend_8_obgc_overview_co2_summer.pdf}} \\ 
         
         \subfloat[pCO$_{\text{2,thermal}}$ trend \text{[ppm/8yrs]}]
        {\label{fig:thermal_summer_neg-a}%
        \includegraphics[scale=1.55,trim=13.2cm 7.3cm 4.5cm 17.785cm,clip]{\membernegative _positive_trend_8_obgc_overview_co2_summer.pdf}} 
        \subfloat[$\Delta$pCO$_{\text{2,non-thermal}}$ trend \text{[ppm/8yrs]}]
        {\label{fig:thermal_summer_neg-b}%
         \includegraphics[scale=1.55,trim=13.2cm 4.5cm 4.4cm 20.65cm,clip]{\membernegative _positive_trend_8_obgc_overview_co2_summer.pdf}}
        \caption{Linear trends in austral summer during the most negative monotonic sea-air 8-year CO$_2$ flux trend: (a) sea-air CO$_2$ flux, (b) $\Delta$pCO$_2$, (c) \ac{SLP} and wind vectors overlain as arrows, (d) \acf{SST}, (e) pCO$_{2,\text{thermal}}$ and (f) $\Delta$pCO$_{2,\text{non-thermal}}$; hatched areas indicate where trends are outside the 5\% significance level.} \label{fig:neg_summer}
\end{figure}



\begin{figure}[h!]
        \myfloatalign
        \subfloat[CO$_2$ flux trend \text{[kgC m$^{-2}$yr$^{-1}$ /8yrs]}]
        {\label{fig:neg_winter-co2flux}%
        \includegraphics[scale=1.55,trim=13.2cm 18.75cm 4.5cm 6.3cm,clip]{\membernegative _positive_trend_8_obgc_overview_co2_winter.pdf}} %\quad
        \subfloat[$\Delta$pCO$_2$ trend \text{[ppm/8yrs]}]
        {\label{fig:neg_winter-dpco2}%
         \includegraphics[scale=1.55,trim=13.2cm 15.9cm 4.5cm 9.225cm,clip]{\membernegative _positive_trend_8_obgc_overview_co2_winter.pdf}} \\
         
         \subfloat[\ac{SLP} and wind trends \text{[hPa/8yrs]}]
        {\label{fig:neg_winter-slp}%
        \includegraphics[scale=1.55,trim=13.2cm 10.2cm 4.5cm 14.9cm,clip]{\membernegative _positive_trend_8_obgc_overview_co2_winter.pdf}} %\quad
        \subfloat[\ac{SST} trend \text{[$^{\circ}$C/8yrs]}]
        {\label{fig:neg_winter-sst}%
         \includegraphics[scale=1.55,trim=13.18cm 13.05cm 4.5cm 12.075cm,clip]{\membernegative _positive_trend_8_obgc_overview_co2_winter.pdf}} \\ 
         
         \subfloat[pCO$_{\text{2,thermal}}$ trend \text{[ppm/8yrs]}]
        {\label{fig:thermal_winter_neg-a}%
        \includegraphics[scale=1.55,trim=13.2cm 7.3cm 4.5cm 17.785cm,clip]{\membernegative _positive_trend_8_obgc_overview_co2_winter.pdf}} 
        \subfloat[$\Delta$pCO$_{\text{2,non-thermal}}$ trend \text{[ppm/8yrs]}]
        {\label{fig:thermal_winter_neg-b}%
         \includegraphics[scale=1.55,trim=13.2cm 4.5cm 4.4cm 20.65cm,clip]{\membernegative _positive_trend_8_obgc_overview_co2_winter.pdf}}
        \caption{Linear trends in austral winter during the most negative monotonic 8-year sea-air CO$_2$ flux trend: (a) sea-air CO$_2$ flux, (b) $\Delta$pCO$_2$, (c) \ac{SLP} and wind vectors overlain as arrows, (d) \acf{SST} (e) pCO$_{2,\text{thermal}}$ and (f) $\Delta$pCO$_{2,\text{non-thermal}}$; hatched areas indicate where trends are outside the 5\% significance level.} \label{fig:neg_winter}
\end{figure}


\clearpage
\section{Misc}
\begin{figure}[h!]
 \myfloatalign
 \captionsetup[subfigure]{labelformat=empty,justification=centering}
	\subfloat[Average phytoplankton growth rate factor \text{[d$^{-1}$]}]{\includegraphics[scale=.66,trim=0.03cm 2cm 0cm 3cm,clip]{limf.png}}
	\caption{Combined light- \& temperature limitation function for the average phytoplankton growth rate in HAMOCC}
\label{fig:lighttemplimf}
\end{figure}
\vspace{.5cm}
\begin{figure}[h!]%bt]
\myfloatalign%\centering
\captionsetup[subfigure]{labelformat=empty,justification=centering}
	\subfloat[pCO$_{\text{2,ocean}}$(Alk, DIC, SST, SSS) \text{[ppm]}]{
	\includegraphics[scale=.63,trim=0.0cm 0cm 0cm 0cm,clip]{dic_alk_pco2_processes.png}}
	\caption{Deffeyes diagram and process contributions}
\label{fig:deffeyes_processes}
\end{figure}

%https://tex.stackexchange.com/questions/85776/change-figure-numbering-for-appendix


\cleardoublepage%*******************************************************
% List of Figures and of the Tables
%*******************************************************
\clearpage
\begingroup 
    \let\clearpage\relax
    \let\cleardoublepage\relax
    \let\cleardoublepage\relax
    %*******************************************************
    % List of Figures
    %*******************************************************    
	\section{\listfigurename}     
    %\phantomsection 
    \refstepcounter{dummy}
    %\addcontentsline{toc}{section}{\listfigurename}
 %   \pdfbookmark[1]{\listfigurename}{lof}
    \listoffigures

    \vspace{8ex}

    %*******************************************************
    % List of Tables
    %*******************************************************
%	\section{\listtablename}     
    %\phantomsection 
 %   \refstepcounter{dummy}
    %\addcontentsline{toc}{section}{\listtablename}
  %  \pdfbookmark[1]{\listtablename}{lot}
  %  \listoftables
        
 %   \vspace{8ex}
%   \newpage
    
    %*******************************************************
    % List of Listings
    %*******************************************************      
%	\section{\lstlistlistingname}     
      %\phantomsection 
 %   \refstepcounter{dummy}
    %\addcontentsline{toc}{section}{\lstlistlistingname}
    %\addcontentsline{toc}{chapter}{\lstlistlistingname}
 %   \pdfbookmark[1]{\lstlistlistingname}{lol}
    %\lstlistoflistings 

 %   \vspace{8ex}
       
    %*******************************************************
    % Acronyms
    %*******************************************************
    %\phantomsection 
%    \refstepcounter{dummy}
    %\pdfbookmark[1]{Acronyms}{acronyms}
    %\markboth{\spacedlowsmallcaps{Acronyms}}{\spacedlowsmallcaps{Acronyms}}
    %\chapter*{Acronyms}
    %\begin{acronym}[UMLX]
    %    \acro{DRY}{Don't Repeat Yourself}
    %    \acro{API}{Application Programming Interface}
    %    \acro{UML}{Unified Modeling Language}
    %\end{acronym}                     
\endgroup


%********************************************************************
% Other Stuff in the Back
%*******************************************************
\cleardoublepage\include{FrontBackmatter/Bibliography}
\cleardoublepage%*******************************************************
% Acknowledgments
%*******************************************************
\pdfbookmark[1]{Acknowledgments}{acknowledgments}

%\begin{flushright}{\slshape    
%    We have seen that computer programming is an art, \\ 
%    because it applies accumulated knowledge to the world, \\ 
%    because it requires skill and ingenuity, and especially \\
%    because it produces objects of beauty.} \\ \medskip
%    --- \defcitealias{knuth:1974}{Donald E. Knuth}\citetalias{knuth:1974} \citep{knuth:1974}
%\end{flushright}

%\let\clearpage\relax
%\let\cleardoublepage\relax
%\let\cleardoublepage\relax
\chapter*{Acknowledgments}

Writing this thesis was an incredible journey of personal learning and failures, which I could not have accomplished without the help of my environment.\newline

I thank my supervisors Tatiana Ilyina for setting the foundations of this thesis and guidance in contents, and Norbert Frank for external feedback and encouraging me to apply to an external research center like \acf{MPI}.\newline

I enjoyed my visit to the \acs{HAMOCC} group in Hamburg. I especially thank Hongmei for high-frequency support, also Irene, Tinka and J{\"o}ran for their time and patience in explaining me the basics of marine biogeochemistry over and over again.

I also acknowledge \acs{MPI} for the inhouse development of the extremely quick and easy-to-use commando-line to CDO \citep{CDO} and Luis Kornblueh for running the historical \acf{MPI-ESM LE}.\newline

Finally, I thank my parents for encouraging me to study what I strived for, financing my studies and for who they are. And Alisha for a great 2017.
\cleardoublepage%*******************************************************
% Declaration
%*******************************************************
\refstepcounter{dummy}
\pdfbookmark[0]{Declaration}{declaration}
\chapter*{Declaration}
\thispagestyle{empty}
\setlength{\parindent}{0em}
\vspace{3\baselineskip}
Erkl\"{a}rung:\par
\vspace{3\baselineskip}
Ich versichere, dass ich diese Arbeit selbstst\"{a}ndig verfasst habe und keine
anderen als die angegebenen Quellen und Hilfsmittel benutzt habe.\par
\vspace{3\baselineskip}

 
\noindent\textit{\myLocation, Juni 2017}

%\smallskip

\begin{flushright}
    \begin{tabular}{m{5cm}}
        \\ \hline
        \centering\myName \\
    \end{tabular}
\end{flushright}
  
%*******************************************************
\end{document}
% ********************************************************************

%\input{../classicthesis-config}

%\begin{document}

%************************************************
\chapter{Introduction}\label{ch:introduction}
%************************************************
%where to cite Landschuetzer2016

%\paragraph{Why the Southern Ocean is important} 
%The global Carbon cycle
The oceans are major carbon sink by taking up about 25-30\% of the anthropogenic carbon emissions from the atmosphere \citep{Sabine2004,Quere2016}. As a key region, the Southern Ocean is estimated to contribute about 50\% to the global ocean carbon sink \citep{Takahashi2012}. Due to the sparse spatial and temporal coverage in the Southern Ocean, various observational CO$_2$ flux products yield large uncertainties \citep{Roedenbeck2015}. Also modeling results have a large spread \citep{Wang2016} and claim the Southern Ocean as a constraint to reduce model uncertainties in future projections \citep{Kessler2016}.\newline

%\paragraph{Southern Ocean observations and demand for models}
Recent observations suggest pronounced decadal variations in the Southern Ocean carbon sink \citep{Roedenbeck2013,landschuetzer2015}. However, due to the sparse spatial and temporal coverage of measurement data, it is challenging to discern the dynamics of internally varying processes, which demands for the evaluation with models. Earth system models (\acs{ESM}s) are a useful tool to analyze processes that contribute to variability.

Forcing an ocean model with atmospheric reanalysis data, \cite{Lovenduski2007,Lovenduski2008} demonstrates that increased upwelling due to stronger and southward shifted westerly winds in the Southern hemisphere cause a decline in the Southern Ocean carbon sink. Yet, \acs{ESM}s, containing a freely evolving coupled atmospheric and ocean component, don't capture the multi-year variations in sea-air CO$_2$ flux as suggest by observations \citep{Wang2016}. Using a large ensemble of simulations with perturbed initial conditions but identical forcing and model allows to separate trends into an ensemble mean trend, the forced signal, and the residual, the internal variability \citep{McKinley2016,McKinley2017}.\newline

%\paragraph{What I do and research questions}
By using a large ensemble simulation based on the Max-Planck-Institute Earth System Model (hereafter \acs{MPI-ESM LE}), I investigate the variability of the oceanic carbon uptake to answer the following research questions: 
\begin{itemize}
\item What is the modeled internal variability of the Southern Ocean carbon sink? 
\item What are the sources of internal variability?
%\item How does variability in biological and physical processes influence the carbon sink?
\item What are the contributions of different processes to multi-year trends in sea-air CO$_2$ flux?
\end{itemize} 


 
%why did Tatiana want me to do this:
%all coupled models fail to reproduce SO variability, ocean-only models with NCEP forcing catch it, so its probably the winds and need coupled ensemble simulation
 
%\paragraph{Working hypothesis}
The \ac{SAM}, characterizing the strength and position of the westerly winds, is known to be the dominant mode of climate variability in the Southern hemisphere \citep{Thompson2000,Thompson2011}. Supposing the strength and position of the westerlies winds as the major reason for climate variability for the Southern Ocean \citep{Thompson2000}, how does the carbon system respond? Changes in westerly winds alter circulation patterns, which directly effect the carbon sink via the thermal pCO$_2$ effect, circulation of carbon and biological production.\newline

%\paragraph{Revisit processes}
Working on my research questions builds the outline for this thesis, in which I revisit the dominant processes leading to extreme trends in the Southern Ocean carbon sink in the biogeochemical model \acs{HAMOCC} [similar \cite{Lovenduski2007,Lovenduski2008}]:

I use a large ensemble of \acs{MPI-ESM} simulations (\autoref{ch:methods}) and evaluate the model in the key features related to the Southern Ocean carbon sink variability (\autoref{ch:eval}). \autoref{ch:trends} focuses the response of individual processes related to changes in Southern hemisphere winds, which are already discussed in the literature, \eg temperature effect \citep{Takahashi1993,Lovenduski2007}, circulation \citep{Abernathey2011,Hauck2013,Lauderdale2016,Lovenduski2008} and biology \citep{Lovenduski2005,Hauck2013,Tagliabue2014}. \autoref{ch:pCO2separation} quantitative assesses these responses on oceanic pCO$_2$. This revisit is particularly interesting as other large ensembles of perturbed initial conditions do not capture strong decadal variations in the Southern Ocean carbon sink; whereas \acs{MPI-ESM LE} does [private communication N. Lovenduski (\acs{NCAR}) and S. Schlunegger (\acs{GFDL}), see section \ref{sec:PICLE} for details]. In chapter \ref{ch:conclusions} I draw my main conclusions and % on the driving processes for strong multi-year sea-air CO$_2$ flux trends and Understanding the response of the carbon system helps to evaluate the strong trends in \acs{MPI-ESM LE} and 
%Furthermore, I
evaluate how relatable perturbed initial conditions large ensembles are for internal variability in observations and give an outlook on possible future research. %obs?!

%thesis outline?

%\end{document}

