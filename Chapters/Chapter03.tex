%************************************************
\chapter{Math Test Chapter}\label{ch:mathtest} % $\mathbb{ZNR}$
%************************************************
Ei choro aeterno antiopam mea, labitur bonorum pri no. His no decore

\section{Some Formulas}
Due to the statistical nature of ionisation energy loss, large
lement\footnote{Examples taken from Walter Schmidt's great gallery: \\
\url{http://home.vrweb.de/~was/mathfonts.html}}.  Continuous processes
such as multiple
scattering and energy loss play a relevant role in the l
section heads. Consider the \texttt{pdfspacing} option.
\begin{equation}
\kappa =\frac{\xi}{E_{\textrm{max}}} %\mathbb{ZNR}
\end{equation}
$E_{\textrm{max}}$ is the maximum transferable energy in a single
collision with an atomic electron.
\[
E_{\textrm{max}} =\frac{2 m_{\textrm{e}} \beta^2\gamma^2 }{1 +
2\gamma m_{\textrm{e}}/m_{\textrm{x}} + \left ( m_{\textrm{e}}
/m_{\textrm{x}}\right)^2}\ ,
\]
where $\gamma = E/m_{\textrm{x}}$, $E$ is energy and
$m_{\textrm{x}}$ the mass of the incident particle,
$\beta^2 = 1 - 1/\gamma^2$ and $m_{\textrm{e}}$ is the electron mass.
$\xi$ comes from the Rutherford scattering cross section
and is defined as:
\begin{eqnarray*} \xi  = \frac{2\pi z^2 e^4 N_{\textrm{Av}} Z \rho
\delta x}{m_{\textrm{e}} \beta^2 c^2 A} =  153.4 \frac{z^2}{\beta^2}
\frac{Z}{A}
  \rho \delta x \quad\textrm{keV},
\end{eqnarray*}
where

\begin{tabular}{ll}
$z$          & charge of the incident particle \\
$N_{\textrm{Av}}$     & Avogadro's number \\
$Z$          & atomic number of the material \\
$A$          & atomic weight of the material \\
$\rho$       & density \\
$ \delta x$  & thickness of the material \\
\end{tabular}
