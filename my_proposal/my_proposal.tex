\documentclass[12pt]{article}

%\usepackage{scicite}

%\usepackage{times}
\usepackage{pdfpages}

%\usepackage[ansinew]{inputenc}
\usepackage{amsmath}
\usepackage{siunitx}
\usepackage[colorlinks=true, urlcolor=blue, linkcolor=blue,citecolor=blue]{hyperref}
%\usepackage[german]{babel}
\usepackage{float}
%\usepackage[numbers]{natbib}
\usepackage[authoryear]{natbib}
%\usepackage[pliannatuthoryear]{natbib}

\usepackage{graphicx}
\graphicspath{{/home/mpim/m300524/MSc_Thesis/gfx/}}

% The following parameters seem to provide a reasonable page setup.

\topmargin 0.0cm
\oddsidemargin 0.2cm
\textwidth 16cm 
\textheight 21cm
\footskip 1.0cm



\renewcommand\refname{References and Notes}

% The following lines set up an environment for the last note in the
% reference list, which commonly includes acknowledgments of funding,
% help, etc.  It's intended for users of BibTeX or the {thebibliography}
% environment.  Users who are hand-coding their references at the end
% using a list environment such as {enumerate} can simply add another
% item at the end, and it will be numbered automatically.

%\newcounter{lastnote}
%\newenvironment{scilastnote}{%
%\setcounter{lastnote}{\value{enumiv}}%
%\addtocounter{lastnote}{+1}%
%\begin{list}%
%{\arabic{lastnote}.}
%{\setlength{\leftmargin}{.22in}}
%{\setlength{\labelsep}{.5em}}}
%{\end{list}}


% Include your paper's title here

\title{\vspace{-2cm} \textbf{MSc-Thesis proposal} \\ \vspace{1cm} The response of the biological pump and upper ocean circulation towards westerly winds dominates the internal variability in the Southern Ocean carbon sink modeled in the coupled MPI-ESM large ensemble simulation.}

\author
{Aaron Spring$^{}$%\\
%\normalsize{$^{1}$Max Planck Institute for Meteorology, Bundesstra{\ss}e 53, 20146 Hamburg, Germany}\\
%\\
}

% Include the date command, but leave its argument blank.

\date{}



%%%%%%%%%%%%%%%%% END OF PREAMBLE %%%%%%%%%%%%%%%%




\begin{document} 

% Double-space the manuscript.

\baselineskip24pt
%\baselineskip18pt

% Make the title.

\maketitle 

%intro SOCS whats new
Observation-based estimates report a large variability in the Southern Ocean carbon sink. There are speculations about the underlying processes because observational data are sparse. The carbon trends are explained by increased upwelling due to intensified winds in the 1990s \citep{LeQuere2007,DeVries2017} and by different carbon uptake responses to changes in circulation in the 2000s \citep{landschuetzer2015}. By analyzing a large ensemble of 100 individually independent simulations with different initial conditions but identical model and forcings (MPI-ESM1.1), I study modeled internal variability. Besides Li and Ilyina [2017, GRL, in prep.] the topic of decadal oceanic carbon uptake trends in a coupled model remains unaddressed. Here I examine the leading mechanisms for the Southern Ocean explicitly.


%what I expect
I target three consecutive questions: 1) How large is the modeled internal variability in the Southern Ocean carbon sink? 2) Can a large ensemble of simulations reproduce similar trends to those observed in the 1990s and 2000s? 3) Which processes drive (decadal) internal variability in our large ensemble?

%how I do it
Explicitly, I scan the large ensemble for somewhat monotonic trends similar to those in the SOM-FFN dataset \citep{landschuetzer2015} in the 1990s and 2000s and identify the dominating processes. %Obviously, decades comprise exactly ten years, but here the process, that triggers the change, matters. 
%I examine the chosen decades with linear trends and EOFs to find correlation and potentially causality. %Here I highlight the process responsible rather than different trend lengths.  %the specific model response happens over a range of trend lengths. %I do not expect longer term (10yr+) trends originating only from internal variability. There is an ongoing debate on the future evolution of primary production under the warming climate.  

%what I know / what I find / doesnt need to be new
Internal variability and decadal trends in the Southern Ocean carbon sink are driven by the strength of the westerly winds. This leads to a two-sided response: to biology and to upper ocean circulation.

%bio
%A first analysis of the ensemble mean and ensemble standard deviation also indicates that at 50-60$^\circ$S, both, the CO$_2$flux and primary production are most variable. %Therefore, I emphasize the response of primary productivity to internally varying processes: 
Phytoplankton growth is sensitive to changes in circulation. Intensifying winds mix the water column deeper than usual. Sea-surface temperature cooling, probably caused by upwelling cold water, facilitates deeper mixing. Thereby standing stock of phytoplankton is transported deeper into the ocean. Decreasing phytosynthetically active radiation at depth reduces the phytoplankton growth rate, the total primary productivity, and hence the carbon sink. 

%upwelling
Intensified westerly winds in the Southern Ocean also strengthen the upper ocean circulation in the Southern hemisphere. Upwelling at 90-50$^\circ$S and Antarctic Intermediate Water formation by downwelling at 50-30$^\circ$S increases. Upwelled water masses from the deep ocean are rich in carbon and hence increase surface pCO$_2$ which weakens the carbon sink. 

%conclusion
Both responses towards intensified winds - biology and circulation - weaken the carbon sink at the latitudes of largest internal variability (50-60$^\circ$S).  
The same mechanisms apply vice versa for decreasing wind strengths leading to positive trends in the carbon sink. 

%einordnung
% our models response compared to obs and other models: iron limitation    
Currently, I am struggling how to arrange my findings into previous literature under the context of iron limitation of phytoplankton growth observed in the Southern Ocean. Also in other model runs, increased upwelling iron supply led to an increase in biological production in the context of intensified winds \citep{Lovenduski2005,Hauck2013,wang2012}. However in our model, iron is abundant in the Southern Ocean, hence iron loses its amplifying leverage on primary production. Therefore, the missing iron limitation in the Southern Ocean in HAMOCC contains two findings: First, a model deficiency could lead to exaggerated internal variability \citep{Nevison2016}. Second, phytoplankton responds in our model only to deeper mixing and not to the implications of nutrient-mixing. 

%\newpage 
\vspace{1cm}

\baselineskip12pt
\bibliography{../Paper/SouthernOceanCarbonSink_new}

\bibliographystyle{abbrvnat}%unsrtnat}%abbrvnat}%plainnat}

\end{document}




















